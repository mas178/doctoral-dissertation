% !TEX program = xelatex
\documentclass[12pt,a4paper,oneside]{report}

\usepackage{fontspec}
\usepackage{xeCJK}
\setCJKmainfont{Hiragino Mincho Pro}[BoldFont=Hiragino Mincho Pro W6]
\usepackage[top=25mm, bottom=20mm, left=20mm, right=20mm]{geometry}
\usepackage{setspace}
\setstretch{1.25}
\usepackage{amsmath} % 数式記号用
\usepackage{xcolor} % 色指定用

% ========================================
% Document Information (ご提示の定義を使用)
% ========================================
\newcommand{\dissertationtitle}{Evolution of Cooperation under Environmental Variability}
\newcommand{\dissertationtitleja}{環境変動下における協力の進化}
\newcommand{\authorname}{稲葉 理晃}
\newcommand{\submissiondate}{2026/01/06}

% ========================================
% Document Body
% ========================================
\begin{document}

\begin{center}
    {\fontsize{16}{18}\selectfont\bfseries 予備審査における講評・コメントへの対応 \par}
\end{center}

\begin{center}
    {\fontsize{16}{18}\selectfont\dissertationtitle \par}
    {\fontsize{14}{16}\selectfont(\dissertationtitleja) \par}
\end{center}

\begin{flushright}
    \submissiondate \authorname \par
\end{flushright}

\vspace{10mm}

\noindent 2025年11月7日に行われた予備審査における講評・コメントに対して以下の通り対応しました。

\vspace{5mm}
\hrule
\vspace{5mm}

\section*{講評者 1}

\subsection*{コメント 1-1}

シミュレーション結果から一般的な知見を主張する場合には、その仕組みを説明することが重要と思います。2章のスライドp.12の説明は明確でよいと思います。3章 (スライドp.23) の調査中の部分についても説得力を強化する分析を期待します。

\subsubsection*{コメント 1-1 への対応}

コメントありがとうございます。
Chapter 3 の「3.2 Results -- Primary drivers of the key results」にて説明しました。

\subsection*{コメント 1-2}

4章は、2章で強調したネットワーク効果が発生しない (相互リンクの強化という要素のない) 環境と思います。
ただ、移動が疑似的なネットワーク効果 (良い相手に隣接して関係を作る・嫌な相手に隣接していたら移動して関係を切る) を生んでいるようにも見えます。
2章の分析と関連させ、環境変動とネットワーク形成の2要素が重要という一貫した主張としてもよいと思います。

\subsubsection*{コメント 1-2 への対応}

ご指摘の通り、2〜4章のモデルにおいて環境変動が協力を促進するメカニズムは、環境変動とネットワーク効果の2要素による一貫したメカニズムで説明できることが分かりました。
Chapter 5 の「5.1 Summary and cross-model comparison -- (i) Why does EV promote cooperation?」にて、説明しました。

\subsection*{コメント 1-3}

環境変動は、その時のエージェントの行動を変えるという意味では突然変異の役割と似ているようにも思います。突然変異にはなく、環境変動に存在する要素で協力行動の形成に重要と思われる要素は論文で議論するとよいと思います。 (行動を変動させる突然変異は既存研究でよく使われていると思いますので、環境変動を考える新規性を際立たせておくのがよいと思います)

\subsubsection*{コメント 1-3 への対応}

突然変異 (mutation) の初出時に、突然変異と環境変動の違いを説明するようにしました。(Ch.2 p.12 -- Reformation)

\vspace{5mm}
\hrule
\vspace{5mm}

\section*{講評者 2}

\subsection*{コメント 2-1}

The base model (スライド#8) でSoR③が仮に左に移動すると② (0.8) はどこに移動するのか、図を見てよくわからなかったので、最終発表でもっと説明があるとよい。

\subsubsection*{コメント 2-1 への対応}

コメントありがとうございます。
論文本体の方にも、``This implies that the entire resource distribution shifts geographically in tandem with the movement of the SoR.'' という説明を追加しました。(Ch.2 p.10 -- EV)

\subsection*{コメント 2-2}

シミュレーションで用いた「公共財ゲーム (PGG)」はなぜなのかの背景を少し説明するとよい。他のゲームを用いる場合は結果が変わるのか?

\subsubsection*{コメント 2-2 への対応}

Pairwise PGG の初出時に、以下の様に説明しました。(Ch.2 pp.11-12 -- Game)

\begin{quote}
We have chosen this game model instead of classic pairwise games to implement the condition that only agents with surplus resources can contribute to other agents.
In classic pairwise games, such as the prisoner's dilemma or the snowdrift game, benefits and costs are fixed at constant values.
This implies that agents would cooperate identically, regardless of resource abundance or scarcity, even under uncertain survival conditions.
This assumption is inconsistent with our research context.
Therefore, we have developed and adopted a novel pairwise PGG model that accounts for resource availability.
\end{quote}

他のゲーム設定、例えば、協力者が全リソースを拠出するゲームや、常に一定量のリソースを拠出するようなゲームでは、協力者が過度に不利となり、協力がほとんど成立しません。

\subsection*{コメント 2-3}

平均協力率の計算をする際に1万世代の後半の5千世代のものを使って計算しているが、その根拠を議論するとよい。また、別の区間を用いた場合、結論が変わるかを少し議論するとよい。

\subsubsection*{コメント 2-3 への対応}

協力率の計算方法の初出時に、以下の説明を追加しました。(Ch.2 p.12 -- Evaluation)

\begin{quote}
The cooperation rate in a single trial does not always reach a steady state within $10,000$ generations and can fluctuate depending on the chosen evaluation window.
However, the aggregate mean over the final $5,000$ generations across $100$ independent trials provides a robust and sufficiently converged measure to identify consistent qualitative patterns.
This approach of utilizing late-stage averages to ensure statistical stability and mitigate initial transient effects is widely adopted in evolutionary simulation research.
\end{quote}

\vspace{5mm}
\hrule
\vspace{5mm}

\section*{講評者 3}

\subsection*{コメント 3-1}

Chapter3のモデルは構造としてはマルチレベルセレクションと近いものがあります。本モデルの特徴と学術的な位置づけを明確にするためにも、一般的なマルチレベルセレクションとの共通点と違いを整理することが有用と思います。

\subsubsection*{コメント 3-1 への対応}

ご指摘ありがとうございます。
以下のパラグラフを追加しました。(Ch.3 p.20 -- 3.1 Model)

\begin{quote}
While Traulsen and Nowak (2006) \textbackslash cite\{Traulsen2006\} is widely recognized as a seminal study on the evolution of cooperation in a 2-level structure, the model presented in this chapter adopts a fundamentally different approach.
Their model primarily analyzes the tension between individual-level interaction favoring defectors ($D$s) and group-level interaction favoring groups with higher proportions of cooperators ($C$s).
In contrast, the 2-level model in this chapter is structured such that the payoff rules of the game (see Subsection \textbackslash nameref\{sec:game\}) inherently define $D$ as the advantageous strategy at both the individual and group levels.
Given this structural disadvantage for $C$s, the primary objective of this chapter is to investigate whether and how the interplay of EV and migration can facilitate the emergence and maintenance of cooperative behavior.
\end{quote}

\subsection*{コメント 3-2}

Chapter4の知見ですが、空間囚人のジレンマの研究では「移動がないこと」が協力のコロニーの生成を可能にするという知見が提出されており、一見すると本モデルとは異なる結果となっています。
既存研究との違いがどのようなメカニズムによって生じるのかを議論することで本研究が適切に既存研究群の上に位置づけられると思います。
Cohen, M. D., Riolo, R. L., \& Axelrod, R. (2001). the Role of Social Structure in
the Maintenance of Cooperative Regimes. Rationality And Society, 13(1), 5-32.
https://doi.org/10.1177/104346301013001001

\subsubsection*{コメント 3-2 への対応}

ご教示頂いた文献を追加し、Chapter 5 の「5.1 Summary and cross-model comparison -- (ii) Why does the impact of migration differ between the 2-level model and the 2D model? および (iii) How does the 2D model relate to the previous literature on cooperation and migration?」にて、議論しました。

\subsection*{コメント 3-3}

また、Chapter3ではレギュラーネットワーク上にグループがあり、その中にエージェントが複数存在するという階層モデルを採用しています。Chapter4では、移動のモデルを扱う際により一般的なモデルとして2次元トーラスに拡張しています。直観的には、2次元トーラス上にグループが存在し、その中に複数のエージェントが存在するという設定のほうが自然な拡張に思われます。エージェントのみを考える非階層構造を採用した動機と妥当性を論文中に明記することで、拡張が妥当かつ自然なものであることを明確にできると思います。

\subsubsection*{コメント 3-3 への対応}

予備審査の説明では誤解を与えてしまったかもしれませんが、Chapter 4 は Chapter 3 の拡張ではありません。
Chapter 4 は先行研究と比較するために、先行研究で多く採用されている2D空間モデルを採用しています。
これらの点を明確にするために、以下の様な説明を追加しました。(Ch.4 p.36)

\begin{quote}
In this chapter, we adopt a 2-dimensional (2D) space, which is widely used in the literature, diverging from the group-structured frameworks of Chapters 2 and 3 to prioritize comparability with prior works.
\end{quote}

\vspace{5mm}
\hrule
\vspace{5mm}

\section*{講評者 4}

\subsection*{コメント 4-1}

Base modelにおいて円環で表された地理的構造は、実際には必須ではないと思うので、一般性を失うように見えるのは損ではないかと思った。

\subsubsection*{コメント 4-1 への対応}

コメントありがとうございます。
ご指摘の通り、円環構造自体は必須ではなく、また現実の地理的構造を網羅する一般的なものではありません。
しかし、Chapter 2 の目的は、地理的隔離が存在する条件下で環境変動が協力に与えるメカニズムをできるだけ基礎的な条件のもとで解明し、研究のスタートラインを提供することにあります。
円環構造は、地理的隔離を表現する最も単純な枠組みであり、境界効果を排除しすべての地域の条件を均一に保つため、妥当な枠組みであると考えています。

Ch.2 では、「our model adopts a highly abstracted approach, aiming to reveal the general effects of EV based on reasonable assumptions while excluding specific details. (Ch.2 p.8)」や「the 1-dimensional spatial structure is used for simplicity (Ch.2 p.9)」と説明しています。

また、「5.3 Limitations and future directions -- Model simplification」にて、以下の様に説明しました。(Ch.5 p.55)

\begin{quote}
The spatial structure was represented by regular topologies, specifically 1-dimensional cycle graphs and 2-dimensional lattices.
These idealized structures are standard in theoretical investigations, as they facilitate the identification of universal evolutionary mechanisms independent of specific geographic idiosyncrasies.
Potential avenues for extending this work include the adoption of more complex network architectures \textbackslash cite\{Wang2015, Masuda2020, Inaba2023\} or the integration of spatially explicit models grounded in paleoenvironmental reconstructions of the MSA in Africa.
\end{quote}

\subsection*{コメント 4-2}

10,000世代のシミュレーションを行うことの適否については、文献の中でよく採用される基準であるのかもしれないが、なぜこれを採用するのかを一言説明してもらえるとありがたかった。

\subsubsection*{コメント 4-2 への対応}

コメント 2-3 にて対応しました。

\subsection*{コメント 4-3}

スライド35に書いているようなモチベーションについては、もっと前のページでより詳しく記述する努力をしてもらえるとより良いと思います。難しいことだとは思いますが、モチベーティング・イグザンプル (中期旧石器時代の"環境"と呼んでいるもの) とモデルの間の関連がよりクリアとなるよう努力をしてもらえたらより良くなると思います。

\subsubsection*{コメント 4-3 への対応}

Chapter 1 の「1.2 Environmental variability and the evolution of modern human behavior」にて説明しました。

\vspace{5mm}
\hrule
\vspace{5mm}

\section*{講評者 5}

\subsection*{コメント 5-1: EV と 協力 の定義}

本論文の二つのキーワード、「EV」と「協力」の定義、及び、変数 ($p_{EV}$など)の
意味が、章ごとに異なるので、章が切り替わる時に読み手が混乱する可能性がある。
環境変動 (EV)の定義が、
\begin{itemize}
\item 第2章:「RV (資源源の移動範囲 $\sigma_R$)」と「UV (閾値の変動 $\sigma_{\theta}$)」の2種類 (p. 8-9)。
\item 第3章:「SoRが1ステップ移動する確率 $p_{EV}$」 (p. 21)。
\item 第4章:「SoRが隣接セルに移動する確率 $p_{EV}$」 (p. 32)。
\end{itemize}
つまり、第2章の $\sigma_R$ は変動の大きさで、第3・4章の $p_{EV}$ は変動(移動)の頻度となっている。
また、協力 (Cooperation)の定義は:
\begin{itemize}
\item 第2章: 「グループ (エージェント)」の戦略 (p. 7)。
\item 第3章: 「グループ」の戦略 かつ 「エージェント」の戦略 (p. 22)。
\item 第4章: 「エージェント」の戦略 (p. 33)。
\end{itemize}
となっている。
例えば、第1章の最後に、本論文で用いる主要概念 (EV, 協力, 移動, 戦略更新) が、各章でどのように異なる形で操作的定義されているかを比較表等であらかじめ一覧化すると良いと思う。

\subsubsection*{コメント 5-1 への対応}

コメントありがとうございます。
Chapter 1. Introduction では、まだモデルについて説明していないので、できるだけモデルの詳細には触れず、概念的な整理を行うようにしました (p.7 Ch.1 -- Table 1.1: Key concepts used throughout this dissertation)。
また、モデルの実装内容については、Appendix にて整理しました (Appendix p.60 -- Table A.1: Detailed comparison of the three simulation models)。

\subsection*{コメント 5-2: 第3章のDiscussion/SummaryのSection}

第3章のDiscussion/SummaryのSectionはこれから追加されると思うが、第2章から拡張された点と新規の知見、つまり、個体レベルの移動を導入したことで協力進化に何が変化したのか、第2章の結果と対比させながら議論し、それが序章の研究質問にどう答えるかを示すと良いと思う。

\subsubsection*{コメント 5-2 への対応}

「Ch.5 Conclusion -- 5.1 Summary and cross-model comparison」の 「Table 5.1: Cross-model comparison of the key features and findings」にて、各モデルの結果を比較整理し、「(i) Why does EV promote cooperation?」にて、メカニズムの共通点と相違点について議論しました。

\subsection*{コメント 5-3: 第3章のグループ間での協力進化について}

第3章のモデルでは、グループ間のゲームとエージェント間のゲームは独立している。従って、第2章のモデルで (1) UVをなくす (2) RVの範囲を1にすると設定をして、さらに、第3章のグループ間ネットワークと第2章のエージェント間ネットワークを揃えると同等のモデルになると思う。このときに、第2章のエージェントと第3章のグループの協力率がほぼ同等になることを確認できると、第2、3章のモデルのつながりが確認できて良いと思う。

\subsubsection*{コメント 5-3 への対応}

仰る通り理論的には第2章と第3章のGroup-Levelは同等のModelであり、ご指摘のような比較ができたほうが良かったのですが、第2章の研究の結果を踏まえて、第3章の実装方法を一部変更したため、直接的な比較は困難となりました。
そこで代わりに、第3章内で、Ablation experiment (p.25) を実施し、Figure 3.2ac (ノーマルな実行結果) と Figure 3.4ac (Individual-level のプロセスを非活性化し、Group-level のプロセスだけを実行した結果) とを比較し、Group-Level の実行結果が変わらないことを確認することで、第2章と第3章のGroup-Levelの同等性を間接的に示しました。

\subsection*{コメント 5-4: タイポなど}

以下、タイポと思われるものをリストアップする。

p.8 「All agents are suited within a geographic structure, …」
suited → situated

p.9 の EV の定義節
before: 「In our study, EV to refers to resource variability, …」
after: 「In our study, EV refers to resource variability, …」

p.9 の AR(1) の説明
before: 「AR(1) process [53–55], $\theta_{t+1} = \mu_\theta(1 - \beta) + \theta_t\beta + \epsilon$」
after: 「first-order autoregressive [AR(1)] process [53–55], $\theta_{t+1} = \mu_\theta(1 - \beta)
+ \beta\theta_t + \epsilon$ ($|\beta|\leq 1$, $\epsilon\sim N(0,\sigma_\theta^2)$)」
初出のAR(1) を正式名称で導入し、母数の範囲と雑音の分布を一行で明示する方が分かりやすい。

p.19
before: 「… their interaction (see Subsection 3.1) and migration (see Subsection
3.1) …」
after: 「… their interaction (see ‘Game’ in Subsection 3.1) and migration (see
‘Migration’ in Subsection 3.1) …」
あるいは、見出しを「3.1.1 Game」「3.1.2 Migration」のように付番する。

p.23
eqn. (3.8) の$Q(j|i)$の右辺ですが、$R_j(k)$は $r_j(k)$ でしょうか?

p.32 「SoR(s) server as simplified representations…」
server → serve

\subsubsection*{コメント 5-4 への対応}

以下の通り修正しました。

\begin{itemize}
    \item Ch.2 (p.9): All agents are situated within a geographic structure, forming an interaction structure.
    \item Ch.2 (p.10): In this dissertation, EV refers to resource variability, represented by stochastic models.
    \item Ch.2 (p.10): The fluctuation is modeled by a first-order autoregressive process, or AR(1) \textbackslash cite\{Hasselmann1976, Vyushin2012, Salcedo2022\}, $\theta_{t+1} = \mu_\theta (1 - \beta) + \theta_t \beta + \epsilon$ with $0 \leq \beta < 1$ and $\epsilon \sim N(0, \sigma_\theta^2)$, where $\mu_\theta$ is the expected value of $\theta$, $\beta$ is the autoregressive coefficient, and $\epsilon$ is a normally distributed noise term with mean $0$ and standard deviation (SD) $\sigma_\theta$.
    \item Ch.3 (p.20): and their interaction (see Subsection \textbackslash nameref\{sec:game\}) and migration (see Subsection  \textbackslash nameref\{sec:migration\})
    \item Ch.3 (p.23): $Q(j|i) = \frac{r_j}{\sum_{\substack{k \neq i \\ r_k > \theta}} r_k}$
    \item Ch.4 (p.37): SoRs serve as simplified representations
\end{itemize}

\vspace{5mm}
\hrule
\vspace{5mm}

\vspace{10mm}
\noindent 以上
\end{document}