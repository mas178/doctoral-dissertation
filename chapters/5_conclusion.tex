\chapter{Conclusion}\label{ch:conclusion}

\section{Summary of results}\label{sec:summary}

This dissertation investigated whether and how environmental variability (EV) promotes the evolution of cooperation through three distinct modeling approaches.

\textcolor{red}{[TODO: Summarize key findings and mechanisms from Chapter 2]}

\textcolor{red}{[TODO: Summarize key findings and mechanisms from Chapter 3]}

\textcolor{red}{[TODO: Summarize key findings and mechanisms from Chapter 4]}

\textcolor{red}{[TODO: Summarize common and divergent findings across the three models, and the reasons]}.

\textcolor{red}{[TODO: Summarize common and divergent mechanisms across the three models, and the reasons]}

\section{Implications and interpretation}\label{sec:implications}

The results of this dissertation have several implications for understanding the evolution of cooperation and human behavior.

First, our research extends the variability selection hypothesis (VSH) beyond individual cognitive adaptations.
EV during the Middle Stone Age (MSA) has been primarily understood as a selective pressure on individual-level cognitive abilities, particularly brain expansion.
Our research demonstrates that EV may have also directly influenced group-level social structures, extending the scope of VSH to encompass collective behavioral patterns.
\textcolor{red}{[TODO: Elaborate this paragraph]}

Second, this dissertation identifies variability selection as a previously underexplored mechanism in cooperation theory.
While numerous mechanisms for the evolution of cooperation have been proposed, variability selection has received limited attention as a potential driver of cooperation.
Our findings demonstrate that variability selection can promote cooperation.
\textcolor{red}{[TODO: Elaborate this paragraph]}

Third, the theoretical insights from this research may inform future empirical work in evolutionary anthropology and archaeology.
\textcolor{red}{[TODO: Elaborate this paragraph]}

\section{Limitations and future directions}\label{sec:limitations}

While this dissertation provides insights into the relationship between EV and cooperation, several limitations warrant acknowledgment and suggest directions for future research.

First, our theoretical findings lack direct empirical validation.
Cooperative behaviors leave limited archaeological traces, making it challenging to test predictions against paleoenvironmental and archaeological data.
Future empirical work integrating paleoclimatic records with archaeological evidence of social organization and resource sharing could test whether cooperation patterns correlate with periods of environmental variability.
\textcolor{red}{[TODO: Elaborate this paragraph.]}

Second, the models employed in this research are too simple as representations of reality.
The spatial structures, migration patterns, interaction mechanisms, and strategy updating rules could be modeled in numerous alternative ways.
While the three modeling approaches serve as appropriate starting points, they are far from exhaustive, and substantial room remains for developing more diverse and realistic models.
\textcolor{red}{[TODO: Elaborate this paragraph.]}

Third, the models are too complex for full analytical treatment.
The three models examined in this dissertation incorporate sufficient detail to capture realistic dynamics but are consequently difficult to analyze mathematically.
Future research could benefit from developing simpler, analytically tractable models that isolate key mechanisms identified in this work.
\textcolor{red}{[TODO: Elaborate this paragraph.]}
