\chapter{Conclusion}\label{ch:5_conclusion}

This dissertation has investigated whether and how environmental variability (EV) promotes the evolution of cooperation using three distinct models presented in Chapters~\ref{ch:2_base_model} through \ref{ch:4_2D_model}.
In this chapter, we begin by synthesizing the results across the models to answer the central research question.
Subsequently, we discuss the broader theoretical implications of these findings and outline potential directions for future research.

\section{Summary and cross-model comparison}\label{sec:5_summary}

This section synthesizes the results and mechanisms from the three models presented in this dissertation, highlighting their commonalities and distinctions.
The central research question underlying this dissertation is whether and how EV facilitates the evolution of cooperation.
Chapter~\ref{ch:2_base_model} examined the effects of EV on cooperation in the absence of migration.
Although migration is a significant factor in the Middle Stone Age (MSA) context, it was excluded in the chapter to isolate the direct impact of EV.
Building on this, Chapter~\ref{ch:3_2Lvl_model} investigated the joint effects of EV and migration on cooperation by introducing migration into the group-structured model.
Finally, Chapter~\ref{ch:4_2D_model} extended the analysis to a 2-dimensional spatial framework, a widely established structure in the literature.
Table~\ref{tab:3_models} provides an overview of the structural features and key findings of the three models.

\begin{table}[!ht]
\centering
\caption{Cross-model comparison of the key features and findings}
\label{tab:3_models}
\begin{tabular}{>{\raggedright\arraybackslash}p{0.19\linewidth}>{\raggedright\arraybackslash}p{0.45\linewidth}>{\raggedright\arraybackslash}p{0.36\linewidth}}
\toprule
Chapter & Model overview & Effects of EV and migration \\
\midrule
\ref{ch:2_base_model} \nameref{ch:2_base_model} &
% Model 1
1-level model (group agents only);\newline
agents on a 1D circular structure;\newline
no migration;\newline
dynamic interaction network via rewiring;\newline
EV types: RV (regional variability by shifting SoR), UV (universal variability by fluctuating threshold), and CV (combined variability of RV and UV). &
% Results 1
(1) RV promotes cooperation.\newline
(2) UV has weak effect.\\
\cmidrule{1-3}
\ref{ch:3_2Lvl_model} \nameref{ch:3_2Lvl_model} &
% Model 2
2-level model (group and individual agents);\newline
groups on a 1D circular structure;\newline
individuals migrate between neighboring groups;\newline
dynamic interaction networks (weight updating) at both levels;\newline
EV by shifting SoR. &
% Results 2
(1) EV promotes cooperation at both levels.\newline
(2) Migration has negligible effect on group-level cooperation.\newline
(3) Moderate migration promotes individual-level cooperation, whereas excessive migration hinders it. \\
\cmidrule{1-3}
\ref{ch:4_2D_model} \nameref{ch:4_2D_model} &
% Model 3
1-level model (individual agents only);\newline
agents on a 2D lattice;\newline
interact with neighbors;\newline
migration based on local resource thresholds;\newline
EV by shifting SoRs. &
% Results 3
(1) EV promotes cooperation (given sufficient migration).\newline
(2) Migration promotes cooperation (given sufficient EV). \\
\bottomrule
\end{tabular}
\end{table}

Beyond this phenomenological comparison, a deeper synthesis is required to uncover the fundamental dynamics governing the evolution of cooperation under EV.
Given that the three models employ distinct spatial structures and interaction rules, the consistent positive effect of EV implies the existence of a robust, model-independent principle.
Conversely, the divergent outcomes regarding migration suggest that its impact is heavily mediated by structural constraints.
To disentangle these factors, the following subsections address three critical questions:
(i) why EV promotes cooperation across the three structurally different models;
(ii) why the impact of migration differs between the 2-level model and the 2D model; and
(iii) how the 2D model relates to the previous literature on cooperation and migration.

\subsection*{(i) Why does EV promote cooperation?}

Despite the structural differences among the three models, a consistent pattern emerged: EV promotes the evolution of cooperation.
The universal mechanism is summarized as follows:
\begin{enumerate}
    \item In static environments, defectors ($D$s) exploit cooperators ($C$s) and establish dominant structures.
    \item EV disrupts this stable dominance by creating stochastic opportunities where either $C$s or $D$s may become resource-rich by chance.
    \item Network effects selectively maintain the dominance of $C$s that emerges accidentally, whereas the dominance of $D$s is not maintained.
\end{enumerate}
This mechanism applies to each model as follows. In the lists below, the numbered steps correspond directly to the three stages outlined above.
\begin{itemize}
    \item Chapter~\ref{ch:2_base_model}~\nameref{ch:2_base_model}:
    \begin{enumerate}
        \item In static environments, $D$s establish dominance.
        \item Regional Variability (RV) fluctuates the strategy distribution, creating opportunities also for $C$s.
        \item The link rewiring mechanism maintains the dominance of $C$s that emerges by chance, whereas the dominance of $D$s is not maintained.    
    \end{enumerate}
    On the other hand, Universal Variability (UV) does not generate sufficient fluctuations in strategy distribution, and thus does not create chance opportunities for $C$s.

    \item Chapter~\ref{ch:3_2Lvl_model}~\nameref{ch:3_2Lvl_model} at the group level:
    \begin{enumerate}
        \item In static environments, $D$s establish dominance.
        \item EV fluctuates the group-level resources and strategy distribution, creating opportunities also for $C$s.
        \item The relationship weight updating mechanism strengthens the cooperative relationships, but not the uncooperative ones.
    \end{enumerate}

    \item Chapter~\ref{ch:3_2Lvl_model}~\nameref{ch:3_2Lvl_model} at the individual level:
    \begin{enumerate}
        \item In static environments with migration, per-capita resources are equally and statically distributed, leading to $D$ dominance.
        \item EV generates resource-rich individuals ($C$s and $D$s), allowing them to remain in their group without migration or strategy updates.
        \item When individuals remain in a group, the relationship weight updating mechanism strengthens cooperative relationships, but not uncooperative ones.
    \end{enumerate}

    \item Chapter~\ref{ch:4_2D_model}~\nameref{ch:4_2D_model}:
    \begin{enumerate}
        \item In static environments with migration, $D$s form large stable clusters.
        \item EV disrupts these stable $D$ clusters, forcing them to migrate or update their strategies.
        \item The network effect corresponds to spatial clustering; mobile $C$s can aggregate and persist, whereas forming clusters provides no benefit to $D$s.
    \end{enumerate}
\end{itemize}

This universal applicability suggests that the evolution of cooperation under EV is not a model-specific artifact but a fundamental phenomenon.
In essence, EV functions as a catalyst that disrupts the static entrenchment of defectors, thereby allowing network mechanisms to retain the cooperative clusters that emerge during these dynamic fluctuations.
This implies that environmental dynamics are as critical as network dynamics in unlocking the evolution of cooperation.

\subsection*{(ii) Why does the impact of migration differ between the 2-level model and the 2D model?}

While EV promotes cooperation universally, the effect of migration varies between the 2-level model (Chapter~\ref{ch:3_2Lvl_model}) and the 2D model (Chapter~\ref{ch:4_2D_model}).
In the 2-level model, excessive migration hinders cooperation, whereas in the 2D model, migration consistently promotes it.
This discrepancy arises from the fundamental difference in the time required to establish cooperative relationships in each model.

In the 2-level model, the formation of cooperative networks relies on the gradual accumulation of interaction weights.
This process requires time.
Excessive migration disrupts this accumulation, as individuals leave their groups before stable cooperative relationships can mature.
Therefore, individuals must remain in the same group for a sufficient duration to foster the strong ties necessary for cooperation.

In contrast, in the 2D model, cooperative relationships are determined solely by spatial adjacency and are established instantaneously.
When cooperators migrate and settle next to each other, they immediately form a functional cooperative relationship without the need for a time-consuming weight-building process.
Thus, migration does not reset the progress of relationship formation but simply rearranges the configuration.

This comparison suggests a general insight: excessive mobility hinders cooperation when the formation of relationships requires time or relies on a history of interaction.

\subsection*{(iii) How does the 2D model relate to the previous literature on cooperation and migration?}

The relationship between migration and cooperation has been a subject of debate in evolutionary game theory.
Classical literature generally posits that mobility hinders cooperation.
Dugatkin and Wilson (1991) \cite{Dugatkin1991} showed that Rover---a mobile All-D agent---can overtake cooperative strategies (e.g., All-C and TFT) if the migration cost is low.
Cohen et al. (2001) \cite{Cohen2001} theoretically demonstrated that stable interaction structure, which they termed ``context preservation'', is necessary for the sustainability of cooperative relationships.
According to this view, migration is detrimental because it disrupts the stable social context required for cooperation.

In contrast, more recent studies have suggested that mobility can facilitate cooperation under specific conditions.
Vainstein et al. (2007) \cite{Vainstein2007}, titled ``Does mobility decrease cooperation?'', demonstrated that in low-density spatial environments, mobility allows isolated cooperators to find each other, thereby promoting the formation of cooperative clusters.
Building on Vainstein's framework, subsequent studies on various research focuses \cite{Cong2012, Chen2012, He2020, Dhakal2020, Ren2021, Yang2023, Zhang2025} have identified conditions under which migration promotes cooperation.

The 2D model (Chapter~\ref{ch:4_2D_model}) also extends Vainstein's framework, focusing on the influence of EV.
In this model, migration consistently promotes cooperation.
This result is consistent with Vainstein's finding that mobility promotes cooperation in low-density spatial environments.
However, the 2D model demonstrates complex population structure dynamics that are not observed in Vainstein's model.
This is because Vainstein's model does not consider EV and lacks spatial heterogeneity, whereas the 2D model explicitly focuses on the influence of EV.

Regarding the classical literature, the results in the 2D model appear to oppose the view that mobility hinders cooperation.
However, this apparent discrepancy arises because the mobility in the 2D model is not random but is triggered by local resource depletion, which does not disrupt the stable formation of cooperative clusters.
Therefore, the results in the 2D model do not contradict the classical literature, as they operate under fundamentally different migration mechanisms.
Notably, in the 2-level model (Chapter~\ref{ch:3_2Lvl_model}), excessive migration hinders cooperation, which is fully consistent with the classical literature.

\section{Implications and significance}\label{sec:5_implications}

The findings of this dissertation carry implications and significance for the theoretical understanding of cooperation, the variability selection hypothesis (VSH) and related debates in human evolution (the cognitive buffer hypothesis (CBH) and the social brain hypothesis (SBH)), and potential applications to broader contexts.

This dissertation advances the theoretical understanding of cooperation by establishing EV as a robust driver of cooperative behavior.
While previous research has examined mechanisms under static conditions, the role of environmental dynamics has remained understudied.
The consistent finding across three structurally distinct models suggests that environmental factors deserve greater attention in theoretical frameworks of cooperation.
The universal mechanism identified in Section \ref{sec:5_summary} reveals a fundamental interplay between environmental dynamics and network dynamics, suggesting that cooperation cannot be fully understood by examining social interactions in isolation from environmental context.
This perspective integrates with, rather than replaces, established cooperation mechanisms.
Specifically, EV does not introduce an entirely new mechanism but rather modulates the effectiveness of existing mechanisms by altering the environmental conditions under which they operate.

The VSH \cite{Potts1996, Potts1998} posits that intensified environmental fluctuations during the MSA in Africa favored versatilists capable of rapid adaptation.
While this hypothesis has been supported by temporal correlations between environmental changes and the emergence of modern human behavior, the causal mechanisms linking EV to specific behavioral traits have remained unclear.
The findings of this dissertation provide theoretical support for one component of this link by demonstrating plausible mechanisms through which EV could have promoted cooperative behavior.

Furthermore, these findings contribute to the ongoing debate between the CBH \cite{Schuck-Paim2008, Sol2008, Sol2009} and the SBH \cite{Whiten1988, Dunbar1998, Barrett2007, Grove2008, Knight2011, Hayes2014, Faith2021, Dunbar2024}.
The CBH proposes that EV selected for enhanced cognitive abilities, providing a theoretical basis for the VSH.
The SBH, often presented as an alternative, argues that social complexity was the primary driver of cognitive evolution.
The present findings suggest that EV may have promoted cooperative behavior---a core element of the social complexity emphasized by the SBH.
If EV drove not only individual cognitive abilities but also cooperative social behavior, the CBH and the SBH may be complementary rather than competing.

While this dissertation is primarily motivated by questions concerning human evolution in the MSA, the findings have broader implications for understanding cooperation in other contexts.
The fundamental mechanism whereby EV promotes cooperation by disrupting defector dominance and enabling cooperator clustering may operate across diverse biological and social systems.
Although direct application to other domains is not straightforward, the identified mechanisms hold potential for shedding light on ecological systems or contemporary human societies facing increased environmental risks due to climate change and other factors.

\section{Limitations and future directions}\label{sec:5_limitations}

While this dissertation provides theoretical support for the role of environmental variability in promoting cooperation, several limitations should be acknowledged.
These limitations also point toward productive directions for future research.

\subsection*{Limitations of model abstraction}

The models presented in this dissertation are deliberately abstract, prioritizing the identification of fundamental mechanisms over the reproduction of specific historical scenarios.
However, this abstraction entails trade-offs.
The spatial structures employed, whether one-dimensional circular arrangements or two-dimensional lattices, represent simplifications of the complex topographies that characterized Middle Stone Age Africa.
Real landscapes feature irregular terrain, varying connectivity between regions, and heterogeneous resource distributions that cannot be fully captured by regular grid structures.
Similarly, the representation of environmental variability through stochastically moving Sources of Resources, while conceptually capturing the unpredictability of resource availability, does not incorporate the full complexity of paleoenvironmental dynamics, including seasonal cycles, multi-year climate oscillations, or cascading ecological effects.

The models also employ a single abstract resource type, whereas real environments present multiple resource categories with different spatial and temporal distributions.
The interactions between these resource types, such as trade-offs between water availability and food sources, or between safety and resource abundance, likely shaped behavioral decisions in ways not captured by the present models.
Extending the framework to incorporate multiple resource dimensions represents an important direction for future work.

\subsection*{Limitations in behavioral representation}

The binary strategy space of cooperation and defection, while analytically tractable, oversimplifies the behavioral repertoire available to real organisms.
Human cooperation in particular exhibits graduated forms, conditional strategies, and context-dependent modulation that cannot be reduced to a simple dichotomy.
The strategy update mechanisms employed in the models, based on imitation of successful neighbors, represent only one of many possible learning rules.
Alternative mechanisms including reinforcement learning, conformist transmission, payoff-biased transmission, and prestige-biased transmission may yield different dynamics under environmental variability.

Furthermore, the models do not incorporate cognitive constraints that would have limited information processing and decision-making in ancestral populations.
The assumption that agents can accurately assess neighbor fitness and optimally choose migration destinations or role models may overestimate the rationality of real decision-makers.
Incorporating bounded rationality and incomplete information would enhance the realism of the models.

\subsection*{Limitations regarding migration}

The migration mechanisms in Chapters 3 and 4 are triggered exclusively by resource scarcity.
While resource-seeking migration is undoubtedly important, human mobility decisions are also influenced by social factors including conflict avoidance, mate-seeking, kinship obligations, and information exchange.
The models do not capture these alternative migration drivers, which may interact with environmental variability in ways not explored here.
The finding that the impact of migration differs between the two-level model and the two-dimensional model suggests that the specific implementation of migration mechanisms significantly affects outcomes, warranting systematic exploration of alternative formulations.

The unit of migration is also abstractly defined, particularly in Chapter 3, where individuals may represent families or small kin groups.
The internal dynamics of these migration units, including decision-making processes and differential impacts on unit members, are not modeled.
Future work could explicitly incorporate household or kin-group structures to examine how intra-unit dynamics interact with environmental variability.

\subsection*{Limitations in evolutionary representation}

This dissertation focuses on cultural evolution rather than biological evolution, treating strategies as culturally transmitted traits subject to social learning rather than genetic inheritance.
While this framing is appropriate for understanding behavioral change on timescales relevant to the Middle Stone Age, it does not address the biological evolution of cognitive capacities that would have enabled sophisticated cooperation.
The relationship between environmental variability, brain evolution, and cooperative capacity remains an open question that requires integration across biological and cultural evolutionary frameworks.

The treatment of groups as units of adaptation, employed particularly in Chapter 2, remains contested in evolutionary biology.
Although the models frame group-level dynamics in terms of cultural rather than genetic evolution, the conceptual challenges associated with group selection apply in modified form to cultural group selection.
The conditions under which cultural group selection can operate effectively, and whether the models presented here satisfy those conditions, merit further theoretical scrutiny.

\subsection*{Absence of empirical validation}

The findings of this dissertation are derived entirely from computational simulations and lack direct empirical validation.
While the models are designed to capture plausible features of Middle Stone Age environments and behaviors, the absence of quantitative calibration to archaeological or paleoenvironmental data limits the specificity of the conclusions.
The models can identify qualitative patterns and mechanisms but cannot predict, for example, the specific magnitude of environmental variability required to produce a given level of cooperation, or the timescales over which cooperative behavior would emerge under particular conditions.

\subsection*{Future directions}

Several promising directions for future research emerge from these limitations.

First, enriching the environmental representation would enhance the realism and applicability of the models.
This includes incorporating multiple resource types with distinct spatial and temporal dynamics, modeling intrinsic environmental variability where human activities feed back to affect resource availability, and calibrating model parameters to paleoenvironmental reconstructions from specific archaeological sites.
Such calibration could enable more specific predictions about the conditions under which cooperation would have been favored in particular regions during particular time periods.

Second, expanding the behavioral strategy space beyond binary cooperation and defection would allow examination of more nuanced cooperative dynamics.
This could include graduated contribution levels, conditional strategies that respond to partner behavior or environmental conditions, and reputation-based strategies that track interaction histories.
Incorporating these richer strategy spaces would enable investigation of how environmental variability affects the evolution of specific cooperative mechanisms such as indirect reciprocity and generalized exchange.

Third, developing analytical approaches to complement the simulation studies would strengthen the theoretical foundation.
While simulations can explore parameter spaces and identify patterns, mathematical analysis can identify critical thresholds, establish stability conditions, and provide general results that hold across model variants.
Techniques from evolutionary game theory, adaptive dynamics, and statistical mechanics could be brought to bear on the questions addressed here.

Fourth, empirical testing of the theoretical predictions represents an essential next step.
Archaeological evidence could potentially test predictions about the co-occurrence of environmental variability, population mobility, and markers of cooperative behavior.
Laboratory experiments with human participants could examine whether individuals adjust cooperative strategies in response to resource variability under controlled conditions.
Natural experiments in contemporary societies facing environmental change could provide additional evidence regarding the relationship between resource variability and cooperative institutions.

Fifth, extending the framework to other contexts would test the generality of the findings.
While this dissertation is motivated by questions concerning human evolution, the fundamental mechanisms may apply to cooperation in other species facing environmental variability, to organizational behavior under resource uncertainty, and to community responses to climate change.
Such extensions would both test the robustness of the theoretical insights and contribute to understanding cooperation in diverse contemporary settings.

Sixth, integrating the environmental variability perspective with other factors known to influence cooperation would provide a more complete picture.
The models presented here focus on the direct effects of environmental variability operating through resource dynamics and migration patterns.
Future work could examine how environmental variability interacts with reputation systems, punishment mechanisms, communication networks, and cultural institutions.
Such integration would move toward comprehensive models that capture the multiple interacting factors shaping the evolution of human cooperation.

Finally, addressing the cognitive dimensions of cooperation under environmental variability remains an open frontier.
The variability selection hypothesis and the social brain hypothesis offer complementary perspectives on human cognitive evolution, and cooperation provides a potential nexus connecting environmental challenges to social complexity to cognitive demands.
Computational models that incorporate cognitive constraints, learning dynamics, and memory limitations could help clarify how environmental variability might have selected for the cognitive capacities underlying human ultrasociality.

Despite the limitations outlined above, this dissertation establishes environmental variability as a robust factor promoting the evolution of cooperation across structurally distinct models.
The consistent findings across different spatial structures, interaction rules, and migration mechanisms suggest that the relationship between environmental variability and cooperation reflects a fundamental principle rather than a model-specific artifact.
Future research building on this foundation can refine the theoretical framework, test its predictions empirically, and extend its applications to diverse contexts where the interplay of environmental dynamics and social behavior shapes evolutionary outcomes.