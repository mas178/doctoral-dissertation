\chapter{Conclusion}\label{ch:5_conclusion}

This dissertation has investigated whether and how environmental variability (EV) promotes the evolution of cooperation using three distinct models presented in Chapters~\ref{ch:2_base_model} through \ref{ch:4_2D_model}.
This chapter synthesizes the results across the models to answer the central research question.
Subsequently, we discuss the broader theoretical implications and significance.
Finally, we acknowledge the limitations of the study and outline potential directions for future research.

\section{Summary and cross-model comparison}\label{sec:5_summary}

This section synthesizes the results and mechanisms from the three models presented in this dissertation, highlighting their commonalities and distinctions.
The central research question underlying this dissertation is whether and how EV facilitates the evolution of cooperation.
Chapter~\ref{ch:2_base_model} examined the effects of EV on cooperation in the absence of migration.
Although migration is a significant factor in the Middle Stone Age (MSA) context, it was excluded in the chapter to isolate the direct impact of EV.
Building on this, Chapter~\ref{ch:3_2Lvl_model} investigated the joint effects of EV and migration on cooperation by introducing migration into the group-structured model.
Finally, Chapter~\ref{ch:4_2D_model} transitioned to a 2-dimensional spatial framework, a widely established structure in the literature.
Table~\ref{tab:3_models} provides an overview of the structural features and key findings of the three models.

\begin{table}[!ht]
\centering
\caption{Cross-model comparison of the key features and findings}
\label{tab:3_models}
\begin{tabular}{>{\raggedright\arraybackslash}p{0.19\linewidth}>{\raggedright\arraybackslash}p{0.45\linewidth}>{\raggedright\arraybackslash}p{0.36\linewidth}}
\toprule
Chapter & Model overview & Effects of EV and migration \\
\midrule
\ref{ch:2_base_model} \nameref{ch:2_base_model} &
% Model 1
1-level model (group agents only);\newline
agents on a 1D circular structure;\newline
no migration;\newline
dynamic interaction network via rewiring;\newline
EV types: RV (regional variability by shifting SoR), UV (universal variability by fluctuating threshold), and CV (combined variability of RV and UV). &
% Results 1
(1) RV promotes cooperation.\newline
(2) UV has weak effect.\newline
(3) CV reflects additive effects of RV and UV.\\
\cmidrule{1-3}
\ref{ch:3_2Lvl_model} \nameref{ch:3_2Lvl_model} &
% Model 2
2-level model (group and individual agents);\newline
groups on a 1D circular structure;\newline
individuals migrate between neighboring groups;\newline
dynamic interaction networks (weight updating) at both levels;\newline
EV by shifting SoR. &
% Results 2
(1) EV promotes cooperation at both levels.\newline
(2) Migration has negligible effect on group-level cooperation.\newline
(3) Moderate migration promotes individual-level cooperation, whereas excessive migration hinders it. \\
\cmidrule{1-3}
\ref{ch:4_2D_model} \nameref{ch:4_2D_model} &
% Model 3
1-level model (individual agents only);\newline
agents on a 2D lattice;\newline
interact with neighbors;\newline
migration based on local resource thresholds;\newline
EV by shifting SoRs. &
% Results 3
(1) EV promotes cooperation when mobility is sufficient.\newline
(2) Migration promotes cooperation in the presence of EV. \\
\bottomrule
\end{tabular}
\end{table}

Beyond this phenomenological comparison, a deeper synthesis is required to uncover the fundamental dynamics governing the evolution of cooperation under EV.
Given that the three models employ distinct spatial structures and interaction rules, the consistent positive effect of EV implies the existence of a robust, model-independent principle.
Conversely, the divergent outcomes regarding migration suggest that its impact is heavily mediated by structural constraints.
To disentangle these factors, the following subsections address three critical questions:
(i) why EV promotes cooperation across the three structurally different models;
(ii) why the impact of migration differs between the 2-level model and the 2D model; and
(iii) how the 2D model relates to the previous literature on cooperation and migration.

\subsection*{(i) Why does EV promote cooperation?}

Despite the structural differences among the three models, a consistent pattern emerged: EV promotes the evolution of cooperation.
The common mechanism is summarized as follows:
\begin{enumerate}
    \item In static environments, defectors ($D$s) exploit cooperators ($C$s) and establish dominant structures.
    \item EV disrupts this stable dominance, generating stochastic opportunities for not only $D$s but also $C$s to become resource-rich.
    \item Network effects---driven by relationship updating or spatial clustering---selectively maintain the dominance of $C$s that emerges stochastically, whereas the dominance of $D$s is not maintained.
\end{enumerate}
This mechanism applies to each model as follows. In the lists below, the numbered steps correspond directly to the three stages outlined above.
\begin{itemize}
    \item Chapter~\ref{ch:2_base_model}~\nameref{ch:2_base_model}:
    \begin{enumerate}
        \item In static environments, $D$s establish dominance.
        \item Regional Variability (RV) fluctuates the strategy distribution, creating opportunities also for $C$s.
        \item The link rewiring mechanism maintains the dominance of $C$s that emerges by chance, whereas the dominance of $D$s is not maintained.
    \end{enumerate}
    On the other hand, Universal Variability (UV) does not generate sufficient fluctuations in strategy distribution, and thus does not create chance opportunities for $C$s.

    \item Chapter~\ref{ch:3_2Lvl_model}~\nameref{ch:3_2Lvl_model} at the group level:
    \begin{enumerate}
        \item In static environments, $D$s establish dominance.
        \item EV fluctuates the group-level resources and strategy distribution, creating opportunities also for $C$s.
        \item The relationship weight updating mechanism strengthens the cooperative relationships, but not the uncooperative ones.
    \end{enumerate}

    \item Chapter~\ref{ch:3_2Lvl_model}~\nameref{ch:3_2Lvl_model} at the individual level:
    \begin{enumerate}
        \item In static environments with migration, per-capita resources are equally and statically distributed, leading to $D$ dominance.
        \item EV generates resource-rich individuals ($C$s and $D$s), allowing them to remain in their group without migration or strategy updates.
        \item When individuals remain in a group, the relationship weight updating mechanism strengthens cooperative relationships, but not uncooperative ones.
    \end{enumerate}

    \item Chapter~\ref{ch:4_2D_model}~\nameref{ch:4_2D_model}:
    \begin{enumerate}
        \item In static environments with migration, $D$s form large stable clusters.
        \item EV disrupts these stable $D$ clusters, forcing them to migrate or update their strategies.
        \item The network effect corresponds to spatial clustering;
        mobile $C$s form clusters and gain benefits through mutual cooperation, whereas $D$s derive no reciprocal advantages from forming their own clusters.
        Consequently, $C$ clusters are self-reinforcing, whereas $D$ clusters are not.        
    \end{enumerate}
\end{itemize}

This consistent applicability suggests that the evolution of cooperation under EV is not a model-specific artifact but a fundamental phenomenon.
In essence, EV functions as a catalyst that disrupts the static entrenchment of defectors, thereby allowing network mechanisms to selectively retain the cooperative structures that emerge during these stochastic fluctuations.
This implies that environmental dynamics are as critical as network dynamics in unlocking the evolution of cooperation.

\subsection*{(ii) Why does the impact of migration differ between the 2-level model and the 2D model?}

While EV promotes cooperation consistently, the effect of migration varies between the 2-level model (Chapter~\ref{ch:3_2Lvl_model}) and the 2D model (Chapter~\ref{ch:4_2D_model}).
In the 2-level model, excessive migration hinders cooperation, whereas in the 2D model, migration consistently promotes cooperation under EV.
This discrepancy arises from the fundamental difference in the time required to establish cooperative relationships in each model.

In the 2-level model, the formation of cooperative networks relies on the gradual accumulation of interaction weights.
This process requires time.
Excessive migration disrupts this accumulation, as individuals leave their groups before stable cooperative relationships can mature.
Therefore, individuals must remain in the same group for a sufficient duration to foster the strong ties necessary for cooperation.

In contrast, in the 2D model, cooperative relationships are determined solely by spatial adjacency and are established instantaneously.
When cooperators migrate and settle next to each other, they immediately form a functional cooperative relationship without the need for a time-consuming weight-building process.
Thus, migration does not reset the progress of relationship formation but simply rearranges the configuration.

This comparison suggests a general insight: excessive mobility hinders cooperation when the formation of relationships requires time or relies on a history of interactions.

\subsection*{(iii) How does the 2D model relate to the previous literature on cooperation and migration?}

The relationship between migration and cooperation has been a subject of debate in evolutionary game theory.
Classical literature generally posits that mobility hinders cooperation.
Dugatkin and Wilson (1991) \cite{Dugatkin1991} showed that Rover---a mobile All-D agent---can overtake cooperative strategies (e.g., All-C and Tit-for-Tat (TFT)) if the migration cost is low.
Cohen et al. (2001) \cite{Cohen2001} theoretically demonstrated that stable interaction structure, which they termed ``context preservation'', is necessary for the sustainability of cooperative relationships.
According to this view, migration is detrimental because it disrupts the stable social context required for cooperation.

In contrast, more recent studies have suggested that mobility can facilitate cooperation under specific conditions.
Vainstein et al. (2007) \cite{Vainstein2007}, titled ``Does mobility decrease cooperation?'', demonstrated that in low-density spatial environments, mobility allows isolated cooperators to find each other, thereby promoting the formation of cooperative clusters.
Building on Vainstein's framework, subsequent studies on various research focuses \cite{Cong2012, Chen2012, He2020, Dhakal2020, Ren2021, Yang2023, Zhang2025} have identified conditions under which migration promotes cooperation.

The 2D model (Chapter~\ref{ch:4_2D_model}) also extends Vainstein's framework, focusing on the influence of EV.
In this model, migration consistently promotes cooperation.
This result is consistent with Vainstein's finding that mobility promotes cooperation in low-density spatial environments.
However, the 2D model demonstrates complex population structure dynamics that are not observed in Vainstein's model.
This is because Vainstein's model does not consider EV and lacks spatial heterogeneity, whereas the 2D model explicitly focuses on the influence of EV.

Regarding the classical literature, the results in the 2D model appear to oppose the view that mobility hinders cooperation.
However, this apparent discrepancy arises because the mobility in the 2D model is not random but is triggered by local resource depletion, which does not disrupt the stable formation of cooperative clusters.
Therefore, the results in the 2D model do not contradict the classical literature, as they operate under fundamentally different migration mechanisms.
Notably, in the 2-level model (Chapter~\ref{ch:3_2Lvl_model}), excessive migration hinders cooperation, which is fully consistent with the classical literature.

\section{Implications and significance}\label{sec:5_implications}

The findings of this dissertation carry implications and significance for the theoretical understanding of cooperation, the variability selection hypothesis (VSH) and related debates in human evolution (the cognitive buffer hypothesis (CBH) and the social brain hypothesis (SBH)), and potential applications to broader contexts.

This dissertation advances the theoretical understanding of cooperation by establishing EV as a robust driver of cooperative behavior.
While previous research has examined mechanisms under static conditions, the role of environmental dynamics has remained understudied.
The consistent finding across three structurally distinct models suggests that environmental factors deserve greater attention in theoretical frameworks of cooperation.
The common mechanism identified in Section \ref{sec:5_summary} reveals a fundamental interplay between environmental dynamics and network dynamics, suggesting that cooperation cannot be fully understood by examining social interactions in isolation from environmental context.
This perspective integrates with, rather than replaces, established cooperation mechanisms.
Specifically, EV does not introduce an entirely new mechanism but rather modulates the effectiveness of existing mechanisms by altering the environmental conditions under which they operate.

The VSH \cite{Potts1996, Potts1998} posits that intensified environmental fluctuations during the MSA in Africa favored versatilists capable of rapid adaptation.
While this hypothesis has been supported by temporal correlations between environmental changes and the emergence of modern human behavior, the causal mechanisms linking EV to specific behavioral traits have remained unclear.
The findings of this dissertation provide theoretical support for one component of this link by demonstrating plausible mechanisms through which EV could have promoted cooperative behavior.

Furthermore, these findings contribute to the ongoing debate between the CBH \cite{Schuck-Paim2008, Sol2008, Sol2009} and the SBH \cite{Whiten1988, Dunbar1998, Barrett2007, Grove2008, Knight2011, Hayes2014, Faith2021, Dunbar2024}.
The CBH proposes that EV selected for enhanced cognitive abilities, providing a theoretical basis for the VSH.
The SBH, often presented as an alternative, argues that social complexity was the primary driver of cognitive evolution.
The present findings suggest that EV may have promoted cooperative behavior---a core element of the social complexity emphasized by the SBH.
If EV drove not only individual cognitive abilities but also cooperative social behavior, the CBH and the SBH may be complementary rather than competing.

While this dissertation is primarily motivated by questions concerning human evolution in the MSA, the findings have broader implications for understanding cooperation in other contexts.
The fundamental mechanism whereby EV promotes cooperation by disrupting defector dominance and enabling cooperator clustering may operate across diverse biological and social systems.
Although direct application to other domains is not straightforward, the identified mechanisms hold potential for shedding light on ecological systems or contemporary human societies facing increased environmental risks due to climate change and other factors.

\section{Limitations and future directions}\label{sec:5_limitations}

While this dissertation provides theoretical insights into the role of EV in the evolution of cooperation, several limitations should be acknowledged.
The limitations can be broadly categorized into three areas: the absence of empirical validation, the simplification of model assumptions, and the complexity of the simulation approach.
These limitations, however, point toward productive directions for future research.

\subsection*{Empirical validation}

The first limitation concerns the insufficiency of empirical data to directly validate the simulation models and results.
This limitation reflects the inherent scarcity of detailed archaeological and paleoenvironmental data from the MSA in Africa.
The models can identify qualitative patterns and mechanisms but cannot predict, for example, the specific magnitude of EV required to produce a given level of cooperation, or the timescales over which cooperative behavior would emerge under particular conditions.
Nevertheless, this very scarcity underscores the value of computational simulations: where direct observation of evolutionary processes is infeasible and empirical data remain fragmentary, simulations provide a principled approach to exploring plausible mechanisms and generating testable predictions.

Moreover, the theoretical predictions generated here may guide future empirical investigations by directing attention to specific patterns.
The same archaeological or paleoenvironmental data may yield different insights depending on whether researchers approach them with hypotheses regarding the relationship between EV and cooperation.
For instance, evidence of long-distance resource transport during the MSA has traditionally been interpreted primarily as an indicator of expanded home ranges or trading capabilities.
With the hypothesis that EV promotes cooperation, however, the same evidence might be re-examined for temporal correlations with periods of intensified climate instability, potentially revealing that such cooperative behaviors emerged specifically during environmentally variable periods.

\subsection*{Model simplification}

The second limitation concerns the intentional simplifications made to the models.
To isolate the fundamental mechanisms through which EV promotes cooperation, we employed several abstractions in each model component: spatial structure, population structure, EV, interaction, migration, and strategy update.

The spatial structure was represented by regular topologies, specifically 1-dimensional cycle graphs and 2-dimensional lattices.
These idealized structures are standard in theoretical investigations, as they facilitate the identification of universal evolutionary mechanisms independent of specific geographic idiosyncrasies.
Potential avenues for extending this work include the adoption of more complex network architectures \cite{Wang2015, Masuda2020, Inaba2023} or the integration of spatially explicit models grounded in paleoenvironmental reconstructions of the MSA in Africa.

The population structure was represented by non-hierarchical structures (homogeneous group- or individual-level agents) or a minimal 2-level hierarchical structure.
Although real-world human social structures exhibit complex fractal organizations \cite{Bird2019}, restricting the model to simple structures represents a standard methodological approach.
This reflects the heuristic that greater complexity diminishes interpretability and tractability while increasing arbitrariness, often obscuring essential mechanisms without significantly altering the results.
It is not inconceivable, however, that introducing greater demographic realism, such as life history stages or a 3-tiered social architecture consisting of individuals, families, and bands, might add some nuances to the evolutionary dynamics.
Furthermore, the current models assume a constant population size; incorporating demographic fluctuations, such as variations in birth and death rates, could significantly impact the long-term stability of cooperation.

Regarding the EV, the stochastic movement of SoRs effectively captures the landscape dynamics and the consequent unpredictability of resource availability, which are characteristic of the MSA in Africa.
While this abstraction provides a robust foundation for future research, subsequent models could incorporate greater environmental realism, for example by calibrating the dynamics to paleoclimatic proxy data from relevant MSA sites.

In terms of interaction, the models adopted a fundamental pairwise game framework restricted to binary strategies of unconditional cooperation or defection.
Given the active discourse within evolutionary game theory, however, numerous extensions are conceivable.
Future research could explore multi-player games, more complex strategic repertoires such as TFT, reputation-based mechanisms, and social norms.
Furthermore, interaction structures and relationship update rules could be refined based on empirical research and laboratory experiments.
Additionally, the unit of interaction---whether it occurs between individuals or groups---warrants deeper investigation to determine how different levels of agency affect the results.

The migration process was triggered by resource scarcity and migration probability, assuming zero migration costs.
However, human mobility decisions in the real world are influenced by social factors and other environmental factors.
Future models could thus incorporate migration costs \cite{Lee2022} and more realistic migration dynamics \cite{Cong2012, Chen2012, He2020, Dhakal2020, Ren2021, Yang2023, Zhang2025}, for example, by accounting for conflict avoidance, mate-seeking, kinship obligations, and predator avoidance.
Another critical dimension is the unit of migration; while the models treat migration as an individual decision, real-world mobility often operates at the level of collective social units rather than individuals.

Finally, the strategy updating process has been extensively examined within the framework of cultural evolution \cite{Ohtsuki2022, Turner2023, Pi2025}.
Within this field, learning mechanisms for strategy updating are primarily categorized into social learning (SL), individual learning (IL), and their interplay.
The models in this dissertation adopted payoff-biased imitation where agents copy the most successful strategies observed within their communication range.
This is a specific form of SL.
Subsequent models could therefore broaden this scope by incorporating other SL mechanisms such as conformity bias, or integrating IL mechanisms such as reinforcement learning.
Furthermore, the fundamental unit of agency warrants further scrutiny, specifically regarding whether adaptation could operate effectively at the level of the group or should be strictly restricted to the individual.

While these strategic simplifications were indispensable for isolating the fundamental mechanisms of cooperation, future research should evaluate the robustness of these findings by systematically integrating the social and environmental complexities inherent in human evolutionary history.

\subsection*{Model complexity}

While the previous section addressed the limitations of model simplification regarding the empirical realities of the MSA, there exists a converse methodological limitation: the models employed in this dissertation are, in another sense, too complex for rigorous analytical solution.
As a constructive approach, our multi-agent simulations allowed for the reproduction of broad qualitative patterns and the identification of a robust mechanism across different model structures.
However, the interplay between EV, agent mobility, and dynamic interaction structures involves high-dimensional, non-linear interactions that resist straightforward mathematical formulation.

Because of this complexity, while we have provided a qualitative description of how EV disrupts defector dominance and allows network effects to retain cooperation, we have not yet formalized these dynamics into a set of closed-form analytical equations.
Consequently, we cannot yet provide precise quantitative predictions, such as the exact mathematical threshold for the magnitude of EV or the specific migration probability required to flip a system from defection to cooperation.
The current simulation-based approach identifies that a mechanism works, but analytical modeling is needed to define the exact logical boundaries of where and why it fails or succeeds.

Future research should therefore aim to bridge this gap by developing ``minimalist'' analytical models.
It may be possible to isolate the relationship between EV and the evolution of cooperation in its purest form by reducing the model complexity and utilizing standard analytical tools in evolutionary game theory, such as replicator dynamics or Markov chain analysis, and more broadly, the framework of non-equilibrium statistical mechanics \cite{Perc2017, Jusup2022}.
Such models would not replace the simulations presented here but would complement them by providing a more rigorous logical foundation.
Combining the constructive insights from multi-agent simulations with the precision of formal mathematical analysis will ultimately lead to a more sophisticated understanding of the mechanisms that drove the evolution of modern human sociality.

\section{Concluding remarks}\label{sec:5_concluding_remarks}

By analyzing three distinct models, this dissertation has established EV as a robust driver of the evolution of cooperation.
This chapter demonstrated that the seemingly distinct mechanisms detailed in Chapters \ref{ch:2_base_model} through \ref{ch:4_2D_model} are underpinned by a common principle: EV facilitates cooperation by disrupting the static dominance of defectors and allowing network mechanisms to stabilize emerging cooperators.
This consistency indicates that the findings are not model-specific artifacts but represent a consistent and robust pattern in the evolution of cooperation.

Inspired by the VSH, which posits that the evolution of human behavior in the MSA was driven by intensified EV, this dissertation demonstrates a theoretical pathway for the development of cooperative sociality.
Beyond its implications for human evolution, this work advances the theoretical understanding of the evolution of cooperation by establishing EV as an integral component of social dynamics.
Furthermore, the identified mechanisms could potentially offer broader implications for the emergence and persistence of cooperative systems in other biological and contemporary social contexts facing increased environmental risks.

Despite these contributions, this research is not without limitations, particularly regarding empirical validation, model simplification relative to real-world situations, and model complexity for obtaining analytical solutions.
These limitations, however, offer promising avenues for future development.
Future work could strive to increase the persuasiveness of these models by validating them against accumulating archaeological and paleo-environmental data.
Simultaneously, exploring simpler models that allow for analytical solutions is expected to clarify the underlying mechanisms with greater mathematical precision.
Such integrated approaches will further elucidate the complex relationship between environmental dynamics and the evolution of cooperation, and by extension, provide deeper insights into the evolution of sociality.
We hope that the groundwork laid by this dissertation will serve as a meaningful starting point for these future intellectual endeavors.
