\chapter{Conclusion}\label{ch:5_conclusion}

This dissertation has investigated whether and how environmental variability (EV) promotes the evolution of cooperation through three distinct models in Chapters~\ref{ch:2_base_model} to \ref{ch:4_2D_model}.
In this chapter, we begin by synthesizing the results across the models to answer the central research question.
Subsequently, we discuss the broader theoretical implications of these findings and outline potential directions for future research.

\section{Summary and cross-model comparison}\label{sec:5_summary}

This section synthesizes the results and mechanisms from the three models presented in this dissertation, highlighting their commonalities and distinctions.
The central research question underlying this dissertation is whether and how EV facilitates the evolution of cooperation.
Chapter~\ref{ch:2_base_model} examined the effects of EV on cooperation in the absence of migration.
Although migration is a significant factor in the Middle Stone Age (MSA) context, it was excluded in the chapter to isolate the direct impact of EV.
Building on this, Chapter~\ref{ch:3_2Lvl_model} investigated the joint effects of EV and migration on cooperation by introducing migration into the group-structured model.
Finally, Chapter~\ref{ch:4_2D_model} extended the analysis to a 2-dimensional spatial framework, a widely established structure in the literature.
Table~\ref{tab:3_models} provides an overview of the structural features and key findings of the three models.

\begin{table}[!ht]
\centering
\caption{Cross-model comparison of the key features and findings}
\label{tab:3_models}
\begin{tabular}{>{\raggedright\arraybackslash}p{0.19\linewidth}>{\raggedright\arraybackslash}p{0.45\linewidth}>{\raggedright\arraybackslash}p{0.36\linewidth}}
\toprule
Chapter & Model overview & Effects of EV and migration \\
\midrule
\ref{ch:2_base_model} \nameref{ch:2_base_model} &
% Model 1
1-level model (group agents only);\newline
agents on a 1D circular structure;\newline
no migration;\newline
dynamic interaction network via rewiring;\newline
EV types: RV (regional variability by shifting SoR), UV (universal variability by fluctuating threshold), and CV (combined variability by RV and UV). &
% Results 1
(1) RV promotes cooperation.\newline
(2) UV has weak effect.\\
\cmidrule{1-3}
\ref{ch:3_2Lvl_model} \nameref{ch:3_2Lvl_model} &
% Model 2
2-level model (group and individual agents);\newline
groups on a 1D circular structure;\newline
individuals migrate between neighboring groups;\newline
dynamic interaction networks (weight updating) at both levels;\newline
EV by shifting SoR. &
% Results 2
(1) EV promotes cooperation at both levels.\newline
(2) Migration has negligible effect on group-level cooperation.\newline
(3) Moderate migration promotes individual-level cooperation, whereas excessive migration hinders it. \\
\cmidrule{1-3}
\ref{ch:4_2D_model} \nameref{ch:4_2D_model} &
% Model 3
1-level model (individual agents only);\newline
agents on a 2D lattice;\newline
interact with neighbors;\newline
migration based on local resource thresholds;\newline
EV by shifting SoRs. &
% Results 3
(1) EV promotes cooperation (given sufficient migration).\newline
(2) Migration promotes cooperation (given sufficient EV). \\
\bottomrule
\end{tabular}
\end{table}

Beyond this phenomenological comparison, a deeper synthesis is required to uncover the fundamental dynamics governing the evolution of cooperation under EV.
Given that the three models employ distinct spatial structures and interaction rules, the consistent positive effect of EV implies the existence of a robust, model-independent principle.
Conversely, the divergent outcomes regarding migration suggest that its impact is heavily mediated by structural constraints.
To disentangle these factors, the following subsections address three critical questions:
(i) why EV promotes cooperation across the three structurally different models;
(ii) why the impact of migration differs between the 2-level model and the 2D model; and
(iii) how the 2D model relates to the previous literature on cooperation and migration.

\subsection*{(i) Why does EV promote cooperation?}

Despite the structural differences among the three models, a consistent pattern emerged: EV promotes the evolution of cooperation.
The universal mechanism is summarized as follows:
\begin{enumerate}
    \item In static environments, defectors ($D$s) exploit cooperators ($C$s) and establish dominant structures.
    \item EV disrupts this stable dominance by creating stochastic opportunities where either $C$s or $D$s may incidentally become resource-rich.
    \item Network effects selectively maintain the dominance of $C$s that emerges by chance, whereas the dominance of $D$s is not maintained.
\end{enumerate}
This mechanism applies to each model as follows. In the lists below, the numbered steps correspond directly to the three stages outlined above.
\begin{itemize}
    \item Chapter~\ref{ch:2_base_model}~\nameref{ch:2_base_model}:
    \begin{enumerate}
        \item In static environments, $D$s establish dominance.
        \item Regional Variability (RV) fluctuates the strategy distribution, creating opportunities also for $C$s.
        \item The link rewiring mechanism maintains the dominance of $C$s that emerges by chance, whereas the dominance of $D$s is not maintained.    
    \end{enumerate}
    On the other hand, Universal Variability (UV) does not generate sufficient fluctuations in strategy distribution, and thus does not create chance opportunities for $C$s.

    \item Chapter~\ref{ch:3_2Lvl_model}~\nameref{ch:3_2Lvl_model} at the group level:
    \begin{enumerate}
        \item In static environments, $D$s establish dominance.
        \item EV fluctuates the group-level resources and strategy distribution, creating opportunities also for $C$s.
        \item The relationship weight updating mechanism strengthens the cooperative relationships, but not the uncooperative ones.
    \end{enumerate}

    \item Chapter~\ref{ch:3_2Lvl_model}~\nameref{ch:3_2Lvl_model} at the individual level:
    \begin{enumerate}
        \item In static environments with migration, per-capita resources are equally and statically distributed, leading to $D$ dominance.
        \item EV generates resource-rich individuals ($C$s and $D$s), allowing them to remain in their group without migration or strategy updates.
        \item When individuals remain in a group, the relationship weight updating mechanism strengthens cooperative relationships, but not uncooperative ones.
    \end{enumerate}

    \item Chapter~\ref{ch:4_2D_model}~\nameref{ch:4_2D_model}:
    \begin{enumerate}
        \item In static environments with migration, $D$s form large stable clusters.
        \item EV disrupts these stable $D$ clusters, forcing them to migrate or update their strategies.
        \item The network effect corresponds to spatial clustering; mobile $C$s can aggregate and persist, whereas forming clusters provides no benefit to $D$s.
    \end{enumerate}
\end{itemize}

This universal applicability suggests that the evolution of cooperation under EV is not a model-specific artifact but a fundamental phenomenon.
In essence, EV functions as a catalyst that disrupts the static entrenchment of defectors, thereby allowing network mechanisms to retain the cooperative clusters that emerge during these dynamic fluctuations.
This implies that environmental dynamics are as critical as network dynamics in unlocking the evolution of cooperation.

\subsection*{(ii) Why does the impact of migration differ between the 2-level model and the 2D model?}

While EV promotes cooperation universally, the effect of migration varies significantly between the Multi-level Model (Chapter~\ref{ch:3_2Lvl_model}) and the 2D Spatial Model (Chapter~\ref{ch:4_2D_model}).
In Chapter~\ref{ch:3_2Lvl_model}, excessive migration hinders cooperation, whereas in Chapter~\ref{ch:4_2D_model}, migration consistently promotes it.
This discrepancy arises from the different functional roles migration plays in each structural context.

In the group-structured model (Chapter~\ref{ch:3_2Lvl_model}), cooperation relies on "boundary maintenance."
High migration rates blur the boundaries between groups, homogenizing the population and diluting the assortativity necessary for group selection to operate.
Therefore, migration acts as a dispersive force that must be kept moderate to maintain group integrity.

In contrast, in the 2D spatial model (Chapter~\ref{ch:4_2D_model}), which lacks explicit group boundaries, migration serves as a mechanism for "spatial segregation" and "escape."
Mobility allows $C$s to physically distance themselves from $D$s and aggregate with other $C$s in resource-rich areas.
Without mobility, $C$s are trapped with neighbors; with mobility, they can self-organize into clusters.
Consequently, in the absence of fixed group boundaries, migration acts as a constructive force that enables the spontaneous formation of cooperative structures.

\subsection*{(iii) How does the 2D model relate to the previous literature on cooperation and migration?}

The adoption of the 2D spatial model in Chapter~\ref{ch:4_2D_model} serves to validate the robustness of our findings against standard literature.
Most existing theoretical studies on spatial games suggest that "viscosity" (low mobility) is key to the evolution of cooperation because it maintains clusters of relatives or cooperators.
However, our results from Chapter~\ref{ch:4_2D_model} challenge this view in the context of EV.
We demonstrated that under significant environmental fluctuations, high mobility is not detrimental but rather beneficial for cooperation.
This alignment between the group-based results (Chapter~\ref{ch:3_2Lvl_model}) and the standard spatial lattice results (Chapter~\ref{ch:4_2D_model}) confirms that the positive effect of EV is not an artifact of specific group assumptions but a robust phenomenon applicable to general spatial dynamics.

\section{Implications and significance}\label{sec:5_implications}

The results of this dissertation have several implications for understanding the evolution of cooperation and human behavior.

First, our research extends the variability selection hypothesis (VSH) beyond individual cognitive adaptations.
EV during the Middle Stone Age (MSA) has been primarily understood as a selective pressure on individual-level cognitive abilities, particularly brain expansion.
Our research demonstrates that EV may have also directly influenced group-level social structures, extending the scope of VSH to encompass collective behavioral patterns.
\textcolor{red}{[TODO: Elaborate this paragraph]}

Second, this dissertation identifies variability selection as a previously underexplored mechanism in cooperation theory.
While numerous mechanisms for the evolution of cooperation have been proposed, variability selection has received limited attention as a potential driver of cooperation.
Our findings demonstrate that variability selection can promote cooperation.
\textcolor{red}{[TODO: Elaborate this paragraph]}

Third, the theoretical insights from this research may inform future empirical work in evolutionary anthropology and archaeology.
\textcolor{red}{[TODO: Elaborate this paragraph]}

\section{Limitations and future directions}\label{sec:5_limitations}

While this dissertation provides insights into the relationship between EV and cooperation, several limitations warrant acknowledgment and suggest directions for future research.

First, our theoretical findings lack direct empirical validation.
Cooperative behaviors leave limited archaeological traces, making it challenging to test predictions against paleoenvironmental and archaeological data.
Future empirical work integrating paleoclimatic records with archaeological evidence of social organization and resource sharing could test whether cooperation patterns correlate with periods of environmental variability.
\textcolor{red}{[TODO: Elaborate this paragraph.]}

Second, the models employed in this research are too simple as representations of reality.
The spatial structures, migration patterns, interaction mechanisms, and strategy updating rules could be modeled in numerous alternative ways.
While the three modeling approaches serve as appropriate starting points, they are far from exhaustive, and substantial room remains for developing more diverse and realistic models.
\textcolor{red}{[TODO: Elaborate this paragraph.]}

Third, the models are too complex for full analytical treatment.
The three models examined in this dissertation incorporate sufficient detail to capture realistic dynamics but are consequently difficult to analyze mathematically.
Future research could benefit from developing simpler, analytically tractable models that isolate key mechanisms identified in this work.
\textcolor{red}{[TODO: Elaborate this paragraph.]}
