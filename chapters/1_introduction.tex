\chapter{Introduction}\label{ch:introduction}

Cooperation is fundamental to human society.
Some forms of cooperation support basic biological survival and reproduction, including cooperative hunting, resource sharing, collective defense against predators, and alloparental care.
Others reflect uniquely human sociality, such as division of labor, gift-giving, exchange, knowledge transmission, and formation of alliances.
The prosperity of \textit{Homo sapiens} would have been impossible without these behaviors.

Yet the evolutionary origins of cooperation are not fully understood and have been actively studied from Darwin's era to the present day.

Darwin's theory of natural selection in \textit{On the Origin of Species} (1859) \cite{Darwin1859} includes the principle that nature favors traits that increase individual fitness---the ability for individuals to survive and reproduce.
However, although cooperative behaviors, particularly altruistic ones, appear to enhance the fitness of others or the group rather than the actor's own fitness, such behaviors are widespread across diverse taxa, from microorganisms to social insects to mammals.
Darwin himself recognized this puzzle \cite{Darwin1871}.

This puzzle has been investigated first within evolutionary biology, and in later years across diverse disciplines including physics, economics, and psychology, collectively forming a research field known as the evolution of cooperation.

Hamilton (1964) \cite{Hamilton1964} introduced the concept of inclusive fitness, which extends the notion of individual fitness to include not only an individual's direct reproductive success but also the effects of the individual's actions on the reproductive success of genetic relatives, weighted by their degree of relatedness.
Building on this concept, he developed the theory of kin selection, which posits that natural selection can favor traits that reduce an individual's direct fitness if they sufficiently increase the fitness of genetic relatives.
Hamilton formalized this idea in Hamilton's rule: cooperation is favored when $rb > c$, where $r$ is the coefficient of genetic relatedness between actor and recipient, $b$ is the fitness benefit to the recipient, and $c$ is the fitness cost to the actor.

Maynard Smith and Price (1973) \cite{Smith1973} introduced the concept of evolutionarily stable strategy (ESS), a strategy that, if adopted by most members of a population, cannot be invaded by any alternative strategy.
This concept provided a game-theoretic framework for analyzing frequency-dependent selection in social interactions.
Maynard Smith (1982) \cite{Smith1982} further developed this approach, establishing evolutionary game theory as a mathematical framework for studying the evolution of behavioral strategies through strategic interactions among individuals.

Axelrod and Hamilton (1981) \cite{Axelrod1981} demonstrated that reciprocal cooperation can evolve and be sustained through repeated interactions between individuals.

More recently, Nowak (2006) \cite{Nowak2006} synthesized these findings and proposed five fundamental mechanisms for the evolution of cooperation: kin selection, direct reciprocity, indirect reciprocity, network reciprocity, and group selection.

\section{Background}\label{sec:background}

Deepening our understanding of the evolutionary origins of modern human behavior is essential for comprehending the nature of humanity and society.
In anthropology and archaeology, ``modern human behavior" refers to traits unique to or primarily associated with Homo sapiens, marked by abstract thinking, symbolic expression, complex planning, and ultrasociality.
These behaviors include language, religion, mythology, art, music, entertainment, humor, altruism, long-distance trade, and the creation of intergroup networks.
Numerous studies concur that these behavioral patterns emerged during the Middle Stone Age (MSA) in Africa\cite{Mcbrearty2000, Henshilwood2003, dErrico2020, Wilkins2021, Bergstrom2021}.
While there is broad consensus on when and where these behaviors originated, the mechanisms driving their emergence remain enigmatic, despite various proposed theories.

For several years, hypotheses \cite{Potts1996, Potts1998, Potts2013, Potts2018, Potts2020, TrauthMaslin2007, TrauthMaslin2010, TrauthMaslin2014, Ziegler2013, Kalan2020, Siepielski2017, Faith2021} attempting to explain the evolution of hominin behavior by focusing on environmental variability (EV) in Africa during the MSA have garnered significant attention.
Among these, Potts’ variability selection hypothesis (VSH) \cite{Potts1996, Potts1998} proposes that intensified environmental change favored ``versatilists" those capable of rapid adaptation to new environments over ``specialists", who adapt to specific environments, or ``generalists", who adapt across a range of environments.
Here, EV encompasses changes in landscape dynamics (such as land-lake oscillations), climate (such as arid-moist climate oscillations), variations in flora and fauna, ultimately leading to the unpredictability of resource availability.
Initially, this hypothesis was supported by a temporal correlation between intensified environmental changes, the replacement of human species, and the increased complexity of cultural artifacts, such as stone tools and ornaments \cite{Faith2021}.
In addition, the cognitive buffer hypothesis (CBH) \cite{Schuck-Paim2008, Sol2008, Sol2009} provides a neuroscientific basis for VSH, and a mathematical model \cite{Grove2011} demonstrates its theoretical feasibility.
The CBH posits that larger brain sizes in animals, including humans, evolved as a buffer against environmental variability, enhancing survival through improved problem-solving and learning abilities.
In contrast, several theories \cite{Navarrete2011, Will2021, Stibel2023} propose that EV and behavioral diversity do not necessarily drive human encephalization.
These theories emphasize the role of social contexts, as suggested by the social brain hypothesis (SBH) \cite{Whiten1988, Dunbar1998, Barrett2007, Grove2008, Knight2011, Hayes2014, Faith2021, Dunbar2024a}, and consider other factors focus such as dietary influences \cite{DeCasien2017, Grabowski2023}.
The SBH argues that human intellectual abilities evolved in response to the selection pressures of complex social environments, which required the effective management of social relationships within and between groups.
Therefore, much remains unknown about the impact of EV on the evolution of cognitive and behavioral traits in hominins.

Our study suggests that VSH, typically explained through the CBH, may also be connected to the SBH, which is generally considered separate from both VSH and CBH.
While complex social environments encompass various factors, what uniquely characterizes human societies is the extensive and sophisticated cooperation observed, including intergroup cooperation and trade, which contrasts with the intragroup cooperation common in many animal societies.
These advanced social behaviors are central to modern human behavior, and understanding their origins requires focusing on social factors that extend beyond individual-level adaptations, such as those proposed in CBH.
Specifically, we demonstrate that EV fosters intergroup cooperation, which may have contributed to the development of complex social structures.

There are several points of concern when using the term ``group."
First, groups within the complex social environment described by the SBH are nested in a series of fractal-structured networks \cite{Bird2019, Dunbar2020, Dunbar2024a}.
As a result, when smaller groups ally and cooperate to form a larger group, whether this cooperation is viewed as intragroup cooperation within the larger group or intergroup cooperation among the smaller groups depends on the level of analysis.
For simplicity, we assume a certain level of grouping and analyze their intergroup cooperation, though this could alternatively be seen as intragroup cooperation from the perspective of a higher-level group.
Furthermore, while treating groups as units of adaptation is highly debated in evolutionary biology \cite{Smith1976, Okasha2001, Eldakar2011}, our focus here is on cultural evolution rather than biological evolution.
In this cultural context, we assume that a group has a degree of autonomy, treating individual relationships and nested group structures as a black box.
Here, autonomy suggests that the basic behavioral patterns for a group regarding which groups it cooperates with or does not are influenced by intergroup interactions and evolve over time.

In the study of the evolution of cooperation, many studies have been conducted within the framework of evolutionary game theory \cite{Axelrod1981, Nowak2006, Szabo2007, Zaggl2014, Perc2017, West2021}, though most assume a stable environment.
Only a limited number of studies consider environmental factors in the evolution of cooperation, and these, typically in biological or physical contexts, focus on aspects such as extrinsic population variability \cite{Brockhurst2007, Miller2015}, variability in game structure \cite{Gokhale2016, Stojkoski2021}, variability in the strength of selection \cite{Assaf2013}, the impact of EV on learning strategies \cite{Borg2012}, and resource pressure \cite{Pereda2017}.
However, these studies do not fully address our research objective of understanding how EVs influences the evolution of cooperation.

\section{Research Objectives}\label{sec:objectives}

Our research thus investigates how the unpredictability of resource acquisition (EV) may drive the evolution of cooperation among geographically dispersed groups, with a focus on the origins of the social aspects that characterize modern human behavior.

\section{Dissertation Organization}\label{sec:organization}

\textcolor{red}{
ToDo: Describe the structure of the dissertation.
}
