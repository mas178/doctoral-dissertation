\chapter{Introduction}\label{ch:1_introduction}

Cooperation is fundamental to human society.
Some forms of cooperation support basic biological survival and reproduction, including cooperative hunting, resource sharing, collective defense against predators, and alloparental care.
Others reflect uniquely human sociality, such as division of labor, gift-giving, exchange, knowledge transmission, and formation of alliances.
The prosperity of \textit{Homo sapiens} would have been impossible without these behaviors.
However, the evolutionary origins of cooperation are not fully understood and have been actively studied from Darwin's era to the present day.

\section{Theoretical background}\label{sec:background}

Darwin's theory of natural selection in \textit{On the Origin of Species} (1859) \cite{Darwin1859} includes the principle that nature favors traits that increase individual fitness---the ability for individuals to survive and reproduce.
However, although cooperative behaviors, particularly altruistic ones, appear to enhance the fitness of others or the group rather than the actor's own fitness, such behaviors are widespread across diverse taxa, from microorganisms to social insects to mammals.
If natural selection favors traits that increase individual fitness and cooperation appears to decrease individual fitness, why is cooperation so ubiquitous?
Darwin himself recognized this paradox \cite{Darwin1871}.

This puzzle was initially studied within evolutionary biology and was later taken up by a wide range of disciplines, including physics, economics, and psychology, eventually giving rise to an interdisciplinary field known as \textit{the evolution of cooperation}.
The following paragraphs review several key studies in the evolution of cooperation, highlighting their main contributions and limitations.

Hamilton (1964) \cite{Hamilton1964a, Hamilton1964b} introduced the concept of inclusive fitness to explain the evolution of altruistic behaviors among genetically related individuals and formalized this insight as Hamilton's rule.
This theoretical framework provides a powerful explanatory principle for cooperation among kin across diverse taxa, from social insects to primates.
However, Hamilton's rule, in its original form, cannot explain altruistic behaviors between non-relatives, which are particularly prevalent in human societies.
Recent attempts have been made to extend Hamilton's rule to general cooperation mechanisms beyond kin relationships, but the validity of these extensions remains debated \cite{Nowak2010, Abbot2011, vanVeelen2025}.

Maynard Smith and Price (1973) \cite{Smith1973} introduced evolutionary game theory, providing a mathematical framework for analyzing how behavioral strategies, including cooperation and defection, spread in populations.
Their approach treats strategies as heritable traits subject to natural selection, allowing researchers to predict which strategies will persist in populations over evolutionary time.
This framework has become fundamental to studying the evolution of cooperation, as it enables formal analysis of how cooperative and selfish strategies compete and coexist.

Axelrod and Hamilton (1981) \cite{Axelrod1981} demonstrated that reciprocal cooperation can evolve among non-relatives through repeated interactions.
Using the iterated prisoner's dilemma game, they showed that simple reciprocal strategies such as Tit-for-Tat, which cooperates initially and then mimics the opponent's previous action, can be evolutionarily successful when individuals interact repeatedly and can recognize their past partners.
This work established direct reciprocity as a fundamental mechanism for the evolution of cooperation.
However, this mechanism requires individuals to recognize each other and remember past interactions, making it applicable primarily to small groups with repeated encounters.
In large-scale societies where interactions are often anonymous or infrequent, alternative mechanisms are needed to explain the prevalence of cooperation.

Addressing these limitations, Nowak and his collaborators advanced research on various mechanisms that promote cooperation, including indirect reciprocity \cite{Nowak1998, Nowak2005}, network reciprocity \cite{Ohtsuki2006}, and group selection \cite{Traulsen2006}.
Building on these studies, Nowak (2006) \cite{Nowak2006} synthesized the theoretical developments in the field, proposing five fundamental mechanisms for the evolution of cooperation: kin selection, direct reciprocity, indirect reciprocity, network reciprocity, and group selection.
This framework provided a comprehensive taxonomy for understanding how cooperation can evolve under different ecological and social conditions.
However, Nowak's synthesis has been criticized as essentially reformulating Hamilton's rule in different contexts, as each of these mechanisms can be understood within the framework of inclusive fitness theory \cite{West2007}.
Moreover, while this taxonomy is useful for categorizing mechanisms, it does not address how these mechanisms interact or which conditions favor one mechanism over another in realistic ecological settings.

These theoretical developments have established fundamental frameworks for understanding cooperation.
We can no longer naively say that cooperation is a mystery.
However, these frameworks remain highly general and abstract.
In recent years, research has increasingly shifted toward examining cooperation under more specific circumstances and mechanisms.
These studies investigate how factors such as reputation, social norms, memory, complex network structures, learning mechanisms, and environmental variability (EV) shape the evolution of cooperation.
Such context-specific approaches complement the general theoretical frameworks and provide insights into the diverse forms of cooperation observed in nature and human societies.

Among these context-dependent factors, we focus on EV for two reasons.
First, there is a hypothesis supported by empirical data suggesting that EV drove the evolution of uniquely human behaviors (\textit{modern human behavior}), including cooperative behavior.
However, the direct causal relationship between EV and cooperation remains unclear.
Second, other factors such as networks, norms, and learning algorithms are actively researched because they directly influence cooperation.
However, EV appears to be less directly related to cooperation, and thus has been studied less than other factors in the field of the evolution of cooperation.
These two points are detailed in the following Sections \ref{sec:ev_and_mhb} and \ref{sec:ev_and_eoc}.

\section{Environmental variability and the evolution of modern human behavior}\label{sec:ev_and_mhb}

Modern human behavior refers to a suite of traits characteristic of \textit{Homo sapiens}, including abstract thinking, symbolic expression, complex planning, language, art, and crucially, large-scale cooperation and ultrasociality.
Numerous studies concur that these behavioral patterns emerged during the Middle Stone Age (MSA) in Africa \cite{Mcbrearty2000, Henshilwood2003, dErrico2020, Wilkins2021, Bergstrom2021}.
While there is broad consensus on when and where these behaviors originated, the mechanisms driving their emergence remain enigmatic despite various proposed theories.

Among various hypotheses proposed to explain these developments, the variability selection hypothesis (VSH), proposed by Potts (1996, 1998) \cite{Potts1996, Potts1998}, suggests that EV was a primary driving force in human evolution.
According to this hypothesis, intensified environmental fluctuations during MSA in Africa favored ``versatilists'', those capable of rapid adaptation to new environments over ``specialists'', who adapt to specific environments, or ``generalists'', who adapt across a range of environments.
In this context, EV encompasses changes in landscape dynamics (such as land-lake oscillations), climate fluctuations (such as arid-moist climate oscillations), variations in flora and fauna, ultimately leading to the unpredictability of resource availability.

Initially, this hypothesis was supported by a temporal correlation between intensified environmental changes, the replacement of human species, and the increased complexity of cultural artifacts, such as stone tools and ornaments \cite{Faith2021}.
In addition, the cognitive buffer hypothesis (CBH) \cite{Schuck-Paim2008, Sol2008, Sol2009} provides a neuroscientific basis for VSH, and a mathematical model \cite{Grove2011} demonstrates its theoretical feasibility.
The CBH posits that larger brain sizes in animals, including humans, evolved as a buffer against EV, enhancing survival through improved problem-solving and learning abilities.
In contrast, several theories \cite{Navarrete2011, Will2021, Stibel2023} propose that EV and behavioral diversity do not necessarily drive human encephalization.
These theories emphasize the role of social contexts, as suggested by the social brain hypothesis (SBH) \cite{Whiten1988, Dunbar1998, Barrett2007, Grove2008, Knight2011, Hayes2014, Faith2021, Dunbar2024a}, and consider other factors such as dietary influences \cite{DeCasien2017, Grabowski2023}.
The SBH argues that human intellectual abilities evolved in response to the selection pressures of complex social environments, which required the effective management of social relationships within and between groups.

While temporal correlations between EV and the emergence of modern human behavior are evident, the causal mechanisms remain debated.
Moreover, since large-scale complex cooperation is a component of modern human behavior, VSH implicitly suggests that EV played some sort of role in the evolution of cooperation.
This lack of direct explanation provides the first motivation of this dissertation.

\section{Environmental variability and the evolution of cooperation}\label{sec:ev_and_eoc}

Most previous studies on the evolution of cooperation have not considered environmental factors, assuming fixed environmental conditions.
Nevertheless, a small but growing number of studies have examined cooperation under EV.

These works on EV can be broadly categorized into extrinsic EV models and intrinsic EV models.
Extrinsic EV can be represented through several models, including fluctuations in the population that the environment can sustain \cite{Brockhurst2007, Miller2015}, stochastic variations in payoff matrices or game rules \cite{Gokhale2016, Stojkoski2021}, variability in learning, and information transmission mechanisms \cite{Borg2012,Assaf2013}.
Differences in resource availability have also been studied \cite{Pereda2017}, though the model does not involve temporal variability but rather compares static scenarios of abundance and scarcity.
Intrinsic EV, in contrast, refers to EV in which interactions in a system modify the environment, and environmental feedback in turn shapes the system's behavior.

While anthropogenic environmental change is increasingly critical in modern societies, such feedback effects would be negligible in the context of MSA.
Therefore, we focus on extrinsic EV rather than intrinsic EV.
This extrinsic EV should capture the unpredictability of resource availability, as explained in Section~\ref{sec:ev_and_mhb}.
Specifically, we examine how increased intensity of EV between abundant and scarce conditions affects evolutionary dynamics, rather than unidirectional shifts from abundance to scarcity or vice versa.
This aspect of EV has not been previously studied, constituting our second motivation.

\section{Research objectives and approach}\label{sec:objectives_and_approach}

Given the research gaps identified above, this dissertation addresses the following key research questions:
(i) Does EV promote cooperation? and
(ii) If so, how does it?

To address these research questions, we adopt a constructive approach.
The constructive approach is a methodology that seeks to understand phenomena by artificially constructing systems that reproduce them and analyzing their behavior.
In the context of this dissertation, we employ multi-agent simulations based on evolutionary game theory to reproduce and examine the evolutionary dynamics of cooperative behavior under EV.

We adopt this simulation-based approach for three reasons.
First, directly observing evolutionary processes is infeasible because they occur over timescales far beyond experimental reach; simulations allow us to observe such dynamics within tractable timeframes.
Second, simulations enable systematic manipulation of variables such as the intensity of EV, which cannot be controlled in natural situations.
Third, simulations allow us to isolate causal relationships by simplifying conditions, whereas in the real world numerous confounding factors mediate the relationship between EV and cooperation.

However, this approach has inherent limitations.
Because our simulations abstract away from the complexity of real systems, they cannot provide precise quantitative predictions, such as the specific magnitude of EV required to produce a given level of cooperation.
Rather, our objective is to reproduce broad qualitative patterns and to identify the mechanisms through which EV can promote or hinder cooperation, providing theoretical insights that complement empirical research.

\section{Organization}\label{sec:organization}

In this chapter, we reviewed the field of the evolution of cooperation, highlighting that while general theoretical frameworks have been well established, research examining cooperation under specific contexts remains an active area of investigation.
We then identified the relationship between cooperation and EV as the central focus of this dissertation and posed two research questions: whether EV promotes cooperation, and if so, how.
Finally, we introduced our simulation-based constructive approach to address these questions and clarified its scope and limitations.

Numerous approaches to modeling the evolution of cooperation under EV are conceivable.
In this dissertation, we consider three abstract models, examined in Chapters \ref{ch:2_base_model}, \ref{ch:3_2Lvl_model}, and \ref{ch:4_2D_model}.
Given this diversity of possible approaches, these models are not exhaustive, yet they provide an important starting point for future theoretical and empirical research.

Chapter \ref{ch:2_base_model} (\nameref{ch:2_base_model}) introduces a base model that examines the evolution of cooperation among geographically distinct groups under EV.
In this model, groups represent sites at which resources may concentrate, such as riverbanks or lakeshores, where human populations naturally gathered.
A limitation of this model is that it does not account for migration between groups.

Chapter \ref{ch:3_2Lvl_model} (\nameref{ch:3_2Lvl_model}) extends the base model by introducing a two-level structure with individual-level migration.
Here, each geographical group, as defined in Chapter \ref{ch:2_base_model}, contains multiple individuals who can move between groups.
This framework allows us to examine how EV influences cooperation when individual migration is considered.
However, while this model is a natural extension of Chapter \ref{ch:2_base_model}, it represents a relatively unique approach within the existing studies on cooperation and migration, limiting comparability with prior works.

Chapter \ref{ch:4_2D_model} (\nameref{ch:4_2D_model}) addresses this limitation by employing a two-dimensional spatial structure, widely used in the literature on cooperation with migration.
In this model, without assuming explicit group structures, as defined in Chapters \ref{ch:2_base_model} and \ref{ch:3_2Lvl_model}, we investigate how EV influences cooperation among individuals moving across a two-dimensional space.

Chapter \ref{ch:5_conclusion} (\nameref{ch:5_conclusion}) synthesizes the findings from the three models in Chapters \ref{ch:2_base_model}, \ref{ch:3_2Lvl_model}, and \ref{ch:4_2D_model} to discuss the commonalities and differences in the mechanisms.
We further consider the implications and significance of this research, as well as its limitations and directions for future work.

Finally, to provide an overall perspective, Table \ref{tab:key_concepts} summarizes the key concepts used throughout this dissertation, although some are formally described in later chapters.

\begin{table}[htbp]
\centering
\caption{Key concepts used throughout this dissertation}
\label{tab:key_concepts}
\begin{tabular}{>{\raggedright\arraybackslash}p{0.18\textwidth}p{0.77\textwidth}}
\toprule
Term & Description \\
\midrule
Cooperation &
This is a long description that will wrap across multiple lines in the second column.
It explains the concept in detail and provides context for understanding how this term is used throughout the dissertation. \\
\addlinespace
Environmental Variability (EV) &
This is another long description that demonstrates how text wraps in the second column.
It provides a comprehensive explanation of the concept, including its various aspects and implications for the research presented in this work. \\
\addlinespace
Migration and Mobility &
This is another long description that demonstrates how text wraps in the second column.
It provides a comprehensive explanation of the concept, including its various aspects and implications for the research presented in this work. \\
\addlinespace
Agent &
This description explains the theoretical framework used in this dissertation.
It covers the mathematical foundations and how it applies to understanding cooperative behaviors in changing environments. \\
\addlinespace
Group &
This description explains the theoretical framework used in this dissertation.
It covers the mathematical foundations and how it applies to understanding cooperative behaviors in changing environments. \\
\addlinespace
Individual &
This description explains the theoretical framework used in this dissertation.
It covers the mathematical foundations and how it applies to understanding cooperative behaviors in changing environments. \\
\bottomrule
\end{tabular}
\end{table}
