\chapter{Introduction}\label{ch:introduction}

Cooperation is fundamental to human society.
Some forms of cooperation support basic biological survival and reproduction, including cooperative hunting, resource sharing, collective defense against predators, and alloparental care.
Others reflect uniquely human sociality, such as division of labor, gift-giving, exchange, knowledge transmission, and formation of alliances.
The prosperity of \textit{Homo sapiens} would have been impossible without these behaviors.
However, the evolutionary origins of cooperation are not fully understood and have been actively studied from Darwin's era to the present day.

\section{Theoretical background}\label{sec:background}

Darwin's theory of natural selection in \textit{On the Origin of Species} (1859) \cite{Darwin1859} includes the principle that nature favors traits that increase individual fitness---the ability for individuals to survive and reproduce.
However, although cooperative behaviors, particularly altruistic ones, appear to enhance the fitness of others or the group rather than the actor's own fitness, such behaviors are widespread across diverse taxa, from microorganisms to social insects to mammals.
If natural selection favors traits that increase individual fitness and cooperation appears to decrease individual fitness, why is cooperation so ubiquitous?
Darwin himself recognized this paradox \cite{Darwin1871}.

This puzzle was initially studied within evolutionary biology and was later taken up by a wide range of disciplines, including physics, economics, and psychology, eventually giving rise to an interdisciplinary field known as \textit{the evolution of cooperation}.
The following paragraphs review several key studies in the evolution of cooperation, highlighting their main contributions and limitations.

Hamilton (1964) \cite{Hamilton1964a, Hamilton1964b} introduced the concept of inclusive fitness to explain the evolution of altruistic behaviors among genetically related individuals and formalized this insight as Hamilton's rule.
This theoretical framework provides a powerful explanatory principle for cooperation among kin across diverse taxa, from social insects to primates.
However, Hamilton's rule, in its original form, cannot explain altruistic behaviors between non-relatives, which are particularly prevalent in human societies.
Recent attempts have been made to extend Hamilton's rule to general cooperation mechanisms beyond kin relationships, but the validity of these extensions remains debated \cite{Nowak2010, Abbot2011, vanVeelen2025}.

Maynard Smith and Price (1973) \cite{Smith1973} introduced evolutionary game theory, providing a mathematical framework for analyzing how behavioral strategies, including cooperation and defection, spread in populations.
Their approach treats strategies as heritable traits subject to natural selection, allowing researchers to predict which strategies will persist in populations over evolutionary time.
This framework has become fundamental to studying the evolution of cooperation, as it enables formal analysis of how cooperative and selfish strategies compete and coexist.

Axelrod and Hamilton (1981) \cite{Axelrod1981} demonstrated that reciprocal cooperation can evolve among non-relatives through repeated interactions.
Using the iterated prisoner's dilemma game, they showed that simple reciprocal strategies such as Tit-for-Tat, which cooperates initially and then mimics the opponent's previous action, can be evolutionarily successful, when individuals interact repeatedly and can recognize their past partners.
This work established direct reciprocity as a fundamental mechanism for the evolution of cooperation.
However, this mechanism requires individuals to recognize each other and remember past interactions, making it applicable primarily to small groups with repeated encounters.
In large-scale societies where interactions are often anonymous or infrequent, alternative mechanisms are needed to explain the prevalence of cooperation.

Addressing these limitations, Nowak and his collaborators advanced research on various mechanisms that promote cooperation, including indirect reciprocity \cite{Nowak1998, Nowak2005}, network reciprocity \cite{Ohtsuki2006}, and group selection \cite{Traulsen2006}.
Building on these studies, Nowak (2006) \cite{Nowak2006} synthesized the theoretical developments in the field, proposing five fundamental mechanisms for the evolution of cooperation: kin selection, direct reciprocity, indirect reciprocity, network reciprocity, and group selection.
This framework provided a comprehensive taxonomy for understanding how cooperation can evolve under different ecological and social conditions.
However, Nowak's synthesis has been criticized as essentially reformulating Hamilton's rule in different contexts, as each of these mechanisms can be understood within the framework of inclusive fitness theory \cite{West2007}.
Moreover, while this taxonomy is useful for categorizing mechanisms, it does not address how these mechanisms interact or which conditions favor one mechanism over another in realistic ecological settings.

These theoretical developments have established fundamental frameworks for understanding cooperation.
We can no longer naively say that cooperation is a mystery.
However, these frameworks remain highly general and abstract.
In recent years, research has increasingly shifted toward examining cooperation under more specific circumstances and mechanisms.
These studies investigate how factors such as reputation, social norms, memory, complex network structures, learning mechanisms, and environmental variability (EV) shape the evolution of cooperation.
Such context-specific approaches complement the general theoretical frameworks and provide insights into the diverse forms of cooperation observed in nature and human societies.

Among these context-dependent factors, we focus on EV for two reasons.
First, there is a hypothesis supported by empirical data suggesting that EV drove the evolution of uniquely human behaviors (\textit{modern human behavior}), including cooperative behavior.
However, the direct causal relationship between EV and cooperation remains unclear.
Second, other factors such as networks, norms, and learning algorithms are actively researched because they directly influence cooperation.
However, EV appears to be less directly related to cooperation, and thus has been studied less than other factors in the field of the evolution of cooperation.
These two points are detailed in the following Sections \ref{sec:ev_and_mhb} and \ref{sec:ev_and_eoc}.

\section{Modern human behavior and environmental variability}\label{sec:ev_and_mhb}

Modern human behavior refers to a suite of traits characteristic of \textit{Homo sapiens}, including abstract thinking, symbolic expression, complex planning, language, art, and crucially, large-scale cooperation and ultrasociality.
Numerous studies concur that these behavioral patterns emerged during the Middle Stone Age (MSA) in Africa\cite{Mcbrearty2000, Henshilwood2003, dErrico2020, Wilkins2021, Bergstrom2021}.
While there is broad consensus on when and where these behaviors originated, the mechanisms driving their emergence remain enigmatic, despite various proposed theories.

Among various hypotheses proposed to explain these developments, the variability selection hypothesis (VSH), proposed by Potts (1996, 1998) \cite{Potts1996, Potts1998}, suggests that environmental unpredictability and variability was a primary driving force in human evolution.
According to this hypothesis, intensified environmental fluctuations during MSA in Africa favored ``versatilists" those capable of rapid adaptation to new environments over ``specialists", who adapt to specific environments, or ``generalists", who adapt across a range of environments.
In this context, EV encompasses changes in landscape dynamics (such as land-lake oscillations), climate (such as arid-moist climate oscillations), variations in flora and fauna, ultimately leading to the unpredictability of resource availability.

Initially, this hypothesis was supported by a temporal correlation between intensified environmental changes, the replacement of human species, and the increased complexity of cultural artifacts, such as stone tools and ornaments \cite{Faith2021}.
In addition, the cognitive buffer hypothesis (CBH) \cite{Schuck-Paim2008, Sol2008, Sol2009} provides a neuroscientific basis for VSH, and a mathematical model \cite{Grove2011} demonstrates its theoretical feasibility.
The CBH posits that larger brain sizes in animals, including humans, evolved as a buffer against EV, enhancing survival through improved problem-solving and learning abilities.
In contrast, several theories \cite{Navarrete2011, Will2021, Stibel2023} propose that EV and behavioral diversity do not necessarily drive human encephalization.
These theories emphasize the role of social contexts, as suggested by the social brain hypothesis (SBH) \cite{Whiten1988, Dunbar1998, Barrett2007, Grove2008, Knight2011, Hayes2014, Faith2021, Dunbar2024a}, and consider other factors such as dietary influences \cite{DeCasien2017, Grabowski2023}.
The SBH argues that human intellectual abilities evolved in response to the selection pressures of complex social environments, which required the effective management of social relationships within and between groups.
Therefore, while temporal correlations between EV and the behavioral innovations in hominins are evident, the causal mechanisms underlying these relationships remain unclear.

\section{Evolution of cooperation and environmental variability}\label{sec:ev_and_eoc}

Large-scale and complex cooperation is also a key component of modern human behavior.
We are interested in whether cooperation emerged in response to EV, as suggested by VSH for other behavioral traits.

Most previous studies on the evolution of cooperation have not considered environmental factors, assuming fixed environmental conditions.
However, a growing number of studies have examined cooperation under environmental variation.
These studies, typically conducted in biological or physical contexts, focus on aspects such as extrinsic population variability \cite{Brockhurst2007, Miller2015}, variability in game structure \cite{Gokhale2016, Stojkoski2021}, variability in the strength of selection \cite{Assaf2013}, the impact of EV on learning strategies \cite{Borg2012}, and resource pressure \cite{Pereda2017}.

However, these studies examine different types of environmental variation from what we investigate.
Our focus is on temporal fluctuations in resource availability---environments where resource levels rise and fall over time.
While the resource pressure study \cite{Pereda2017} addresses resource levels, it compares static scenarios of abundance versus scarcity rather than examining temporal variation.

\section{Research objectives and approach}\label{sec:objectives_and_approach}

The key questions of this dissertation are:
(i) Does EV promote the evolution of cooperation?
(ii) If so, how does EV promote the evolution of cooperation?

As discussed earlier, EV during the MSA manifested through various interconnected fluctuations that collectively created unpredictable resource availability.
These environmental fluctuations can be modeled in numerous ways.
As a starting point, we consider three distinct approaches to modeling EV.
We begin with a base model that examines the evolution of cooperation among spatially distinct groups.
We then develop two extensions that incorporate migration between groups, a fundamental characteristic of pre-sedentary human populations.
Each approach examines different aspects of how environmental conditions may have influenced cooperation dynamics.

By answering these questions, we seek to understand whether the evolution of large-scale cooperation in humans can be explained, at least in part, by the variability selection framework.

Our study suggests that VSH, typically explained through the CBH, may also be connected to the SBH, which is generally considered separate from both VSH and CBH.
While complex social environments encompass various factors, what uniquely characterizes human societies is the extensive and sophisticated cooperation observed, including intergroup cooperation and trade, which contrasts with the intragroup cooperation common in many animal societies.
These advanced social behaviors are central to modern human behavior, and understanding their origins requires focusing on social factors that extend beyond individual-level adaptations, such as those proposed in CBH.
Specifically, we demonstrate that EV fosters intergroup cooperation, which may have contributed to the development of complex social structures.

\section{Organization}\label{sec:organization}

This chapter (Chapter 1) reviews the field of the evolution of cooperation, highlighting that while general theoretical frameworks have been well established, research examining cooperation under specific contexts remains an active area of investigation.
We then introduce VSH, which proposes that EV drove the evolution of human behavior, including cooperation.
Although temporal correlations between EV and behavioral innovations are evident, the causal mechanisms underlying these relationships remain unclear.
Finally, we present the research objectives of this dissertation and describe three distinct approaches to modeling EV, each designed to examine different aspects of how EV may have influenced cooperation dynamics.

Chapter 2 introduces a base model that examines the evolution of cooperation among geographically distinct groups under EV.
In this model, groups represent sites where resources may concentrate, such as riverbanks or lakeshores, where human populations naturally gathered.
A limitation of this model is that it does not account for migration between groups.

Chapter 3 extends the base model by introducing a two-level structure with individual-level migration.
Here, each geographical group, as defined in Chapter 2, contains multiple individuals who can move between groups.
This framework allows us to examine how EV influences cooperation when individual migration is considered.
However, while this model is a natural extension of Chapter 2, it represents a relatively unique approach within the existing studies on cooperation and migration, limiting comparability with prior works.

Chapter 4 addresses this limitation by employing a two-dimensional spatial structure, widely used in existing studies of cooperation with migration.
In this model, without assuming explicit group structures, as defined in Chapters 2 and 3, we investigate how EV influences cooperation among individuals moving across a two-dimensional space.

Finally, Chapter 5 synthesizes the results from the three models in Chapters 2, 3, and 4 to discuss how EV influences the evolution of cooperation.
We further consider the implications and significance of this research, as well as its limitations and directions for future work.
