\chapter{Introduction}\label{ch:introduction}

Cooperation is fundamental to human society.
Some forms of cooperation support basic biological survival and reproduction, including cooperative hunting, resource sharing, collective defense against predators, and alloparental care.
Others reflect uniquely human sociality, such as division of labor, gift-giving, exchange, knowledge transmission, and formation of alliances.
The prosperity of \textit{Homo sapiens} would have been impossible without these behaviors.
However, the evolutionary origins of cooperation are not fully understood and have been actively studied from Darwin's era to the present day.

\section{Theoretical background}\label{sec:background}

Darwin's theory of natural selection in \textit{On the Origin of Species} (1859) \cite{Darwin1859} includes the principle that nature favors traits that increase individual fitness---the ability for individuals to survive and reproduce.
However, although cooperative behaviors, particularly altruistic ones, appear to enhance the fitness of others or the group rather than the actor's own fitness, such behaviors are widespread across diverse taxa, from microorganisms to social insects to mammals.
If nature favors a trait that increases individual fitness and cooperation decreases individual fitness, why is cooperation so ubiquitous?
Darwin himself recognized this paradox \cite{Darwin1871}.

This puzzle has been investigated first within evolutionary biology, and in later years across diverse disciplines including physics, economics, and psychology, collectively forming a research field known as the evolution of cooperation.
In the following, we review several key studies in the evolution of cooperation, highlighting their main contributions and limitations.

Hamilton (1964) \cite{Hamilton1964a, Hamilton1964b} introduced the concept of inclusive fitness to explain the evolution of altruistic behaviors among genetically related individuals and formalized this insight as Hamilton's rule.
This theoretical framework provides a powerful explanatory principle for cooperation among kin across diverse taxa, from social insects to primates.
However, Hamilton's rule, in its original form, cannot explain altruistic behaviors between non-relatives, which are particularly prevalent in human societies.
Recent attempts have been made to extend Hamilton's rule to general cooperation mechanisms beyond kin relationships, but the validity of these extensions remains debated \cite{Nowak2010, Abbot2011, vanVeelen2025}.

Maynard Smith and Price (1973) \cite{Smith1973} introduced evolutionary game theory, providing a mathematical framework for analyzing how behavioral strategies, including cooperation and defection, spread in populations.
Their approach treats strategies as heritable traits subject to natural selection, allowing researchers to predict which strategies will persist in populations over evolutionary time.
This framework has become fundamental to studying the evolution of cooperation, as it enables formal analysis of how cooperative and selfish strategies compete and coexist.

Axelrod and Hamilton (1981) \cite{Axelrod1981} demonstrated that reciprocal cooperation can evolve among non-relatives through repeated interactions.
Using the iterated prisoner's dilemma, they showed that simple reciprocal strategies such as Tit-for-Tat, which cooperates initially and then mimics the opponent's previous action, can be evolutionarily successful, when individuals interact repeatedly and can recognize their past partners.
This work established direct reciprocity as a fundamental mechanism for the evolution of cooperation.
However, this mechanism requires individuals to recognize each other and remember past interactions, making it applicable primarily to small groups with repeated encounters.
In large-scale societies where interactions are often anonymous or infrequent, alternative mechanisms are needed to explain the prevalence of cooperation.

Addressing these limitations, Nowak and his collaborators advanced research on various mechanisms that promote cooperation, including indirect reciprocity \cite{Nowak1998, Nowak2005}, network reciprocity \cite{Ohtsuki2006}, and group selection \cite{Traulsen2006}.
Building on these studies, Nowak (2006) \cite{Nowak2006} synthesized the theoretical developments in the field, proposing five fundamental mechanisms for the evolution of cooperation: kin selection, direct reciprocity, indirect reciprocity, network reciprocity, and group selection.
This framework provided a comprehensive taxonomy for understanding how cooperation can evolve under different ecological and social conditions.
However, Nowak's synthesis has been criticized as essentially reformulating Hamilton's rule in different contexts, as each of these mechanisms can be understood within the framework of inclusive fitness theory \cite{West2007}.
Moreover, while this taxonomy is useful for categorizing mechanisms, it does not address how these mechanisms interact or which conditions favor one mechanism over another in realistic ecological settings.

These theoretical developments have established fundamental frameworks for understanding cooperation.
We can no longer naively say that cooperation is a mystery.
However, these frameworks remain highly general and abstract.
Applying them to specific ecological and social contexts often requires additional assumptions about cognitive abilities, interaction structures, and environmental conditions.
In recent years, research has increasingly shifted toward examining cooperation under more specific circumstances and mechanisms.
These studies investigate how factors such as memory constraints, complex network structures, social norms, environmental variability, and learning mechanisms shape the evolution of cooperation.
Such context-specific approaches complement the general theoretical frameworks and provide insights into the diverse forms of cooperation observed in nature and human societies.
Among these context-specific factors, this dissertation focuses on environmental variability.

\section{Evolution of cooperation and environmental variability}\label{sec:eoc_and_ev}

Deepening our understanding of the evolutionary origins of modern human behavior is essential for comprehending the nature of humanity and society.
In anthropology and archaeology, ``modern human behavior" refers to traits unique to or primarily associated with Homo sapiens, marked by abstract thinking, symbolic expression, complex planning, and ultrasociality.
These behaviors include language, religion, mythology, art, music, entertainment, humor, altruism, long-distance trade, and the creation of intergroup networks.
Numerous studies concur that these behavioral patterns emerged during the Middle Stone Age (MSA) in Africa\cite{Mcbrearty2000, Henshilwood2003, dErrico2020, Wilkins2021, Bergstrom2021}.
While there is broad consensus on when and where these behaviors originated, the mechanisms driving their emergence remain enigmatic, despite various proposed theories.

For several years, hypotheses \cite{Potts1996, Potts1998, Potts2013, Potts2018, Potts2020, TrauthMaslin2007, TrauthMaslin2010, TrauthMaslin2014, Ziegler2013, Kalan2020, Siepielski2017, Faith2021} attempting to explain the evolution of hominin behavior by focusing on environmental variability (EV) in Africa during the MSA have garnered significant attention.
Among these, Potts’ variability selection hypothesis (VSH) \cite{Potts1996, Potts1998} proposes that intensified environmental change favored ``versatilists" those capable of rapid adaptation to new environments over ``specialists", who adapt to specific environments, or ``generalists", who adapt across a range of environments.
Here, EV encompasses changes in landscape dynamics (such as land-lake oscillations), climate (such as arid-moist climate oscillations), variations in flora and fauna, ultimately leading to the unpredictability of resource availability.
Initially, this hypothesis was supported by a temporal correlation between intensified environmental changes, the replacement of human species, and the increased complexity of cultural artifacts, such as stone tools and ornaments \cite{Faith2021}.
In addition, the cognitive buffer hypothesis (CBH) \cite{Schuck-Paim2008, Sol2008, Sol2009} provides a neuroscientific basis for VSH, and a mathematical model \cite{Grove2011} demonstrates its theoretical feasibility.
The CBH posits that larger brain sizes in animals, including humans, evolved as a buffer against environmental variability, enhancing survival through improved problem-solving and learning abilities.
In contrast, several theories \cite{Navarrete2011, Will2021, Stibel2023} propose that EV and behavioral diversity do not necessarily drive human encephalization.
These theories emphasize the role of social contexts, as suggested by the social brain hypothesis (SBH) \cite{Whiten1988, Dunbar1998, Barrett2007, Grove2008, Knight2011, Hayes2014, Faith2021, Dunbar2024a}, and consider other factors focus such as dietary influences \cite{DeCasien2017, Grabowski2023}.
The SBH argues that human intellectual abilities evolved in response to the selection pressures of complex social environments, which required the effective management of social relationships within and between groups.
Therefore, much remains unknown about the impact of EV on the evolution of cognitive and behavioral traits in hominins.

\section{Research objectives and contributions}\label{sec:objectives_and_contributions}

Our study suggests that VSH, typically explained through the CBH, may also be connected to the SBH, which is generally considered separate from both VSH and CBH.
While complex social environments encompass various factors, what uniquely characterizes human societies is the extensive and sophisticated cooperation observed, including intergroup cooperation and trade, which contrasts with the intragroup cooperation common in many animal societies.
These advanced social behaviors are central to modern human behavior, and understanding their origins requires focusing on social factors that extend beyond individual-level adaptations, such as those proposed in CBH.
Specifically, we demonstrate that EV fosters intergroup cooperation, which may have contributed to the development of complex social structures.

There are several points of concern when using the term ``group."
First, groups within the complex social environment described by the SBH are nested in a series of fractal-structured networks \cite{Bird2019, Dunbar2020, Dunbar2024a}.
As a result, when smaller groups ally and cooperate to form a larger group, whether this cooperation is viewed as intragroup cooperation within the larger group or intergroup cooperation among the smaller groups depends on the level of analysis.
For simplicity, we assume a certain level of grouping and analyze their intergroup cooperation, though this could alternatively be seen as intragroup cooperation from the perspective of a higher-level group.
Furthermore, while treating groups as units of adaptation is highly debated in evolutionary biology \cite{Smith1976, Okasha2001, Eldakar2011}, our focus here is on cultural evolution rather than biological evolution.
In this cultural context, we assume that a group has a degree of autonomy, treating individual relationships and nested group structures as a black box.
Here, autonomy suggests that the basic behavioral patterns for a group regarding which groups it cooperates with or does not are influenced by intergroup interactions and evolve over time.

In the study of the evolution of cooperation, many studies have been conducted within the framework of evolutionary game theory \cite{Axelrod1981, Nowak2006, Szabo2007, Zaggl2014, Perc2017, West2021}, though most assume a stable environment.
Only a limited number of studies consider environmental factors in the evolution of cooperation, and these, typically in biological or physical contexts, focus on aspects such as extrinsic population variability \cite{Brockhurst2007, Miller2015}, variability in game structure \cite{Gokhale2016, Stojkoski2021}, variability in the strength of selection \cite{Assaf2013}, the impact of EV on learning strategies \cite{Borg2012}, and resource pressure \cite{Pereda2017}.
However, these studies do not fully address our research objective of understanding how EVs influences the evolution of cooperation.

Our research thus investigates how the unpredictability of resource acquisition (EV) may drive the evolution of cooperation among geographically dispersed groups, with a focus on the origins of the social aspects that characterize modern human behavior.

\section{Methodological framework}\label{sec:mas}

\section{Organization}\label{sec:organization}

This chapter (Chapter 1) presents the background, open questions, and the objectives and contributions of the dissertation, situating our problem within the study of cooperation under environmental variability (EV).

Chapter 2 formulates a baseline model to examine how uncertainty in resource acquisition affects the evolution of cooperation among geographically dispersed groups. We specify the assumptions, dynamics, and equilibrium concepts, and analyze how key parameters shape the conditions for the emergence and persistence of cooperation.

Chapter 3 extends the framework to a two-level model coupling within-group and between-group interactions. Incorporating learning, imitation, and migration, we study how social hierarchy influences the stability and diversity of cooperative strategies. Alongside analytical results, we provide numerical experiments that chart phase diagrams and transition properties across the parameter space.

Chapter 4 introduces a two-dimensional spatial model with local interactions to assess how spatiotemporal variation in resources impacts network structure and the diffusion of cooperation. Using multi-agent simulations, we evaluate how correlation length, fluctuation intensity, and environmental update speed affect the emergence, maintenance, and breakdown of cooperation, and we examine robustness.

Chapter 5 concludes by synthesizing the findings and discussing how the evolution of cooperation in variable environments may connect to the development of human cognition and social structure. We relate the results to theoretical, empirical, and archaeological evidence, and outline limitations, future directions, and potential applications.
