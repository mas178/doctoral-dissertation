\section*{Summary of Chapter 1: Introduction}

Cooperation is fundamental to human society.
Although the prosperity of \textit{Homo sapiens} would have been impossible without cooperative behaviors, the evolutionary origins of cooperation are not fully understood and have been actively studied since the era of Darwin, leading to the emergence of the interdisciplinary field known as \textit{the evolution of cooperation}.

While established theoretical frameworks exist in the literature on the evolution of cooperation, they remain highly general and abstract.
Consequently, recent research has shifted toward context-specific factors such as reputation, social norms, complex network structures, and environmental variability (EV).
Among these factors, this dissertation focuses on EV for two reasons.
First, evidence suggests that EV may have driven the evolution of uniquely human behaviors (\textit{modern human behavior}) including large-scale and complex cooperative behaviors, although the direct causal relationship has yet to be established.
Second, EV has been studied less than other factors because it appears to be less directly related to cooperation.

For these reasons, this dissertation addresses the following key research questions:
(i) Does EV promote cooperation?
(ii) If so, how does it?
In other words, this dissertation aims to answer whether and how EV promotes the evolution of cooperation.

To address these research questions, this study adopts a constructive approach using multi-agent simulations based on evolutionary game theory to reproduce and analyze the evolutionary dynamics of cooperation under EV.
This simulation-based methodology enables the observation of evolutionary processes over timescales beyond experimental reach, the systematic manipulation of variables such as EV intensity, and the isolation of causal relationships through simplified conditions.
However, because this approach abstracts from the complexity of real systems, it cannot provide precise quantitative predictions.
Instead, the objective of this dissertation is to identify broad qualitative patterns and underlying mechanisms to provide theoretical insights that complement empirical research.

This dissertation considers three abstract models to explore the dynamics of cooperation under EV, as follows:
\begin{itemize}
    \item Chapter 2: The base model
    \item Chapter 3: The 2-level model with migration
    \item Chapter 4: The 2-dimensional (2D) model with migration
\end{itemize}
These models aim to reproduce qualitative patterns and identify fundamental mechanisms through which EV promotes cooperation.
The simulation model consists of the following components: agents, spatial structure, EV, interaction structure, interaction, migration, and strategy updating.
While these components are defined specifically within each model, they share a fundamental architecture as follows:
\begin{itemize}
    \item Each agent has either a cooperative or a non-cooperative strategy.
    \item Spatial structure establishes the topological framework that defines agent placement and the spatial resource gradients.
    \item EV is usually implemented through the stochastic movement of the \textit{Source of Resources} (SoR) across the spatial structure, driving dynamic shifts in resource distribution.
    \item Interactions are conducted within a game-theoretic framework according to the interaction structure, leading to the increase or decrease in resource levels.
    \item Migration and strategy updating function as adaptive mechanisms in response to environmental pressure, where the probability of their execution increases as an agent's resource levels diminish.
    Note that migration is not incorporated in the base model presented in Chapter 2.
\end{itemize}

\section*{Summary of Chapter 2: The base model}

This chapter models geographically separated groups arranged on a 1-dimensional circular structure without migration.
Migration, a critical adaptive response to EV during the Middle Stone Age (MSA), is intentionally excluded to isolate the direct effects of EV.
Here, agents are defined as geographically separated regional groups.
EV is implemented through three specific forms of resource variability: regional variability (RV), involving the spatial shifting of the SoR, universal variability (UV), involving global fluctuations in the resource threshold, and combined variability (CV), combining both RV and UV.
Interaction between agents is governed by a pairwise public goods game (PGG), coupled with a reformation process that triggers strategy updates and network rewiring when resources fall below the resource threshold.

The results show that RV strongly promotes cooperation, UV has a marginal effect, and CV reflects the additive effects of both RV and UV.
This difference arises from two key factors: the fluctuations in strategy distribution generated by EV, and the coevolution of cooperation and network structure.
Both factors work effectively under RV, whereas neither factor operates effectively under UV due to insufficient fluctuations and disrupted network heterogeneity.
These findings provide a theoretical baseline for investigating cooperation under EV with migration in Chapter 3.

\section*{Summary of Chapter 3: The 2-level model with migration}

This chapter extends the base model to investigate the joint effects of EV and migration by introducing a 2-level hierarchical structure consisting of group-level agents and individual-level agents within them.
While the group agents are the same as in Chapter 2, the individual agents newly introduced in this chapter are allowed to migrate between neighboring groups in response to resource scarcity.
Interactions at both group and individual levels are governed by independent weighted networks, where the weights represent the strength of relationships.
Pairwise PGGs are played on both levels and the network weights are updated based on game outcomes; specifically, relationships between cooperators strengthen, while relationships involving defectors weaken or remain static.

The results demonstrate the following key findings:
(i) the group-level cooperation rate gradually increases with EV;
(ii) the individual-level cooperation rate sharply increases even at low levels of EV and plateaus as variability increases further;
(iii) unlike group-level cooperation, individual-level cooperation is strongly affected by migration: moderate levels of migration maximize cooperation, while excessive migration suppresses it.
At both levels, EV promotes cooperation through the formation of cooperative networks that provide robustness against resource fluctuations, migration, and strategy updates, similar to the mechanism in the base model.
In addition, lower initial relationship strength and a faster rate of relationship weight updates further facilitate the formation and maintenance of cooperative networks.
These findings demonstrate that EV remains a powerful driver of cooperation even when individual mobility is incorporated.

\section*{Summary of Chapter 4: The 2-dimensional model with migration}

While the 2-level model in Chapter 3 is a natural extension of the base model, its unique structure limits direct comparison with previous studies on cooperation and migration.
To address this limitation, this chapter adopts a standard 2D lattice space without explicit group structures, a framework widely used in the literature.
The model incorporates unpredictable EV by implementing SoRs that move randomly across the 2D space, generating dynamic spatial heterogeneity in resource availability.
Agents accumulate resources through cooperative or competitive interactions, and the agents with lower resource levels are more likely to migrate to neighboring cells and update their strategies.

The results demonstrate the following key findings:
(i) with sufficient agent mobility, even modest levels of EV promote cooperation; however, further variability does not enhance cooperation;
(ii) agent mobility promotes cooperation in the presence of EV;
(iii) these effects occur because EV disrupts a small number of large stable defector groups that form in resource-rich areas, and agent mobility enables the formation of numerous small cooperator groups at those sites.
These findings demonstrate that EV and agent mobility jointly promote cooperation by preventing fixed defector structures and encouraging the formation of cooperator groups for survival.

\section*{Summary of Chapter 5: Conclusion}

This dissertation has investigated whether and how EV promotes the evolution of cooperation using three distinct models presented in Chapters 2 through 4.
This chapter synthesizes the results across these models to answer the central research question.
Subsequently, we discuss the broader theoretical implications and significance.
Finally, we acknowledge the limitations of the study and outline potential directions for future research.

The core results indicate that EV, implemented as stochastic SoR movement, consistently promotes cooperation across all three structurally distinct models; however, the impact of migration differs between the 2-level model and the 2D model.

The common mechanism through which EV promotes cooperation is as follows:
In static environments, defectors exploit cooperators and establish dominant structures.
However, EV disrupts this stable dominance, generating stochastic opportunities for not only defectors but also cooperators to become resource-rich.
Subsequently, network effects---driven by relationship updating or spatial clustering---selectively maintain the dominance of cooperators that emerges stochastically, whereas the dominance of defectors is not maintained.
This interplay implies that environmental dynamics are as critical as network dynamics in unlocking the evolution of cooperation.

The divergent impact of migration, on the other hand, stems from the time required to establish cooperative relationships within each structural framework.
In the 2-level model, forming cooperative networks depends on the gradual accumulation of interaction weights, which excessive migration disrupts before stable ties can mature.
In contrast, relationships in the 2D model are established instantaneously through spatial adjacency, allowing migration to facilitate the formation of cooperator clusters without a time-consuming weight-building process.
This comparison suggests a general insight: excessive mobility hinders cooperation when the formation of cooperative relationships requires time or relies on a history of interaction.

Regarding the implications and significance of these findings, this dissertation demonstrates a theoretical pathway for the development of cooperative sociality driven by EV during the MSA.
Beyond the archaeological and anthropological context, this work advances the theoretical understanding of the evolution of cooperation by establishing EV as an integral component of social dynamics.
Furthermore, the identified mechanisms could potentially offer broader implications for the emergence and persistence of cooperative systems in other biological and contemporary social contexts facing increased environmental risks.

Despite these contributions, this research is not without limitations, particularly regarding empirical validation, model simplification relative to real-world situations, and model complexity for obtaining analytical solutions.
These limitations, however, offer promising avenues for future development.
Future work could strive to increase the persuasiveness of these models by validating them against accumulating archaeological and paleo-environmental data.
Simultaneously, exploring simpler models that allow for analytical solutions is expected to clarify the underlying mechanisms with greater mathematical precision.
Such integrated approaches will further elucidate the complex relationship between environmental dynamics and the evolution of cooperation, and by extension, provide deeper insights into the evolution of sociality.
We hope that the groundwork laid by this dissertation will serve as a meaningful starting point for these future intellectual endeavors.
